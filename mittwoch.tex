
% time: Wednesday 10:30
% URL: https://pretalx.com/fossgis2020/talk/JV7GRQ/

%
\newTimeslot{10:30}
\noindent\abstractHSAnatomie{%
  Wolfgang Hinsch%
}{%
  Die Welt als runde Sache auf der ebenen Karte%
}{%
}{%
  Es ist nicht möglich, die Erde verzerrungsfrei auf einer Karte darzustellen. Es wird gezeigt, mit
  welchem Abbildungssystem OSM arbeitet, welche Vor- und Nachteile es hat, und eine Auswahl
  relevanter alternativer Koordinatenbezugssysteme aufgezeigt. Dabei werden Begriffe wie
  Referenzellipsoid, Geoid, EPSG, WGS84, NHN etc. vorgestellt.%
}%


%%%%%%%%%%%%%%%%%%%%%%%%%%%%%%%%%%%%%%%%%%%

% time: Wednesday 11:00
% URL: https://pretalx.com/fossgis2020/talk/QMK3JN/

%
\newTimeslot{11:00}
\noindent\abstractHSAnatomie{%
  Hanna Krüger%
}{%
  Ehrenamt im FOSSGIS e.V.%
}{%
}{%
  Was ist eigentlich der FOSSGIS e.V. und was ist sein Ziel, wie funktioniert er und wie sieht das
  Vereinsleben aus?%
}%


%%%%%%%%%%%%%%%%%%%%%%%%%%%%%%%%%%%%%%%%%%%

% time: Wednesday 11:25
% URL: https://pretalx.com/fossgis2020/talk/EZAY3D/

%
\newTimeslot{11:25}
\noindent\abstractHSAnatomie{%
  Marco Lechner%
}{%
  Was ist Open Source?%
}{%
}{%
  Der Vortrag stellt die Geschichte der Entwicklung von Open Source vor und geht auf wichtige
  Grundlagen ein.
  Ziel des FOSSGIS e.V. und der OSGeo ist die Förderung und Verbreitung freier Geographischer
  Informationssysteme (GIS) im Sinne Freier Software und Freier Geodaten. Dazu zählen auch
  Erstinformation und Klarstellung von typischen Fehlinformationen über Open Source und Freie
  Software, die sich über die Jahre festgesetzt haben.%
}%


%%%%%%%%%%%%%%%%%%%%%%%%%%%%%%%%%%%%%%%%%%%

% time: Wednesday 11:40
% URL: https://pretalx.com/fossgis2020/talk/UEWFWR/

%
\newTimeslot{11:40}
\noindent\abstractHSAnatomie{%
  Thomas Skowron%
}{%
  Was ist OpenStreetMap?%
}{%
}{%
  Was ist OpenStreetMap? Die Enstehung, die heutige Bedeutung und die vielfältigen Nutzungsmöglichkeiten der offenen Weltkarte OpenStreetMap werden vorgestellt. Es werden Möglichkeiten zur Mitwirkung präsentiert.%
}%

%%%%%%%%%%%%%%%%%%%%%%%%%%%%%%%%%%%%%%%%%%%

% time: Wednesday 13:00
% URL: https://pretalx.com/fossgis2020/talk/TDNXQF/

%
\newTimeslot{13:00}
\noindent\abstractHSRundbau{%
}{%
  Eröffnung%
}{%
}{%
  Feierliche Eröffnung der Konferenz durch Vertreter des FOSSGIS e.V. mit wichtigen Hinweisen
  zu Ablauf und Organisation.%
}%


%%%%%%%%%%%%%%%%%%%%%%%%%%%%%%%%%%%%%%%%%%%

% time: Wednesday 13:45
% URL: https://pretalx.com/fossgis2020/talk/HEE3SU/

%
\newTimeslot{13:45}
\noindent\abstractHSRundbau{%
  Guillaume Rischard%
}{%
  Wie man offene Geodaten bekommt, jetzt!%
}{%
 Meine Erfahrung in Luxemburg, in der EU und au{\ss}erhalb%
}{%
}%


%%%%%%%%%%%%%%%%%%%%%%%%%%%%%%%%%%%%%%%%%%%

% time: Wednesday 15:00
% URL: https://pretalx.com/fossgis2020/talk/GPMCKV/

%
\newTimeslot{15:00}
\noindent\abstractHSAnatomie{%
  Pirmin Kalberer%
}{%
  2700 interaktive thematische Karten~-- Ein Fall für Vektortiles!%
}{%
}{%
  Die Webkarten des neuen Vogelatlas der Schweizer Vogelwarte bieten dank Vektortile-Technologie
  hohe Interaktivität bei geringem Resourcenbedarf. Der Vortrag zeigt die technischen Hintegründe,
  aber auch viele Karten!%
}%


%%%%%%%%%%%%%%%%%%%%%%%%%%%%%%%%%%%%%%%%%%%

% time: Wednesday 15:00
% URL: https://pretalx.com/fossgis2020/talk/ZFQNNN/

%

\noindent\abstractHSRundbau{%
  Peter Heidelbach%
}{%
  Von ArcGis nach QGIS%
}{%
}{%
  Für die Konvertierung von ArcGIS-Projekten in QGIS-Projekte gibt es derzeit verschiedene Ansätze.
  Die australische Firma North Road entwicklet zur Zeit ein Tool zum Reverse-Engineering der
  Binärdateien. GeoCats Bridge priorisiert dagegen den Export der ArcGIS-Layer als Web-Services. An
  einer nativen Toolbox arbeitet die WhereGroup, welche Projektdaten als QGIS-XML exportiert. Der
  Vortrag beschreibt die unterschiedlichen Vorgehensweisen und stellt Vor- und Nachteile der einzelnen
  Verfahren gegenüber.%
}%


%%%%%%%%%%%%%%%%%%%%%%%%%%%%%%%%%%%%%%%%%%%

% time: Wednesday 15:00
% URL: https://pretalx.com/fossgis2020/talk/YQGPNJ/

%

\noindent\abstractWeismannhaus{%
  Christoph Hormann%
}{%
  Weniger ist mehr~-- zur Auswahl darzustellender Elemente in der digitalen Kartographie%
}{%
}{%
  Die Auswahl von dem, was man in einer Karte darstellt, ist von entscheidender Bedeutung für deren
  Lesbarkeit.  In Karten auf Grundlage von OpenStreetMap-Daten wird diese Auswahl für viele wichtige
  Elemente durch subjektive Klassifizierungen oder die Größe subjektiver Beschriftungs-Geometrien
  vorgenommen, welche eigentlich in OpenStreetMap nichts zu tun haben.  Dieser Vortrag gibt einen
  Überblick über Ansätze zur Bewertung der Bedeutung von Objekten für den Zweck der Auswahl bei der
  Darstellung.%
}%


%%%%%%%%%%%%%%%%%%%%%%%%%%%%%%%%%%%%%%%%%%%

% time: Wednesday 15:30
% URL: https://pretalx.com/fossgis2020/talk/MYTFXF/

%
\newTimeslot{15:30}
\noindent\abstractHSAnatomie{%
  Falk Zscheile%
}{%
  Kartenherstellung zwischen Lizenzen, Daten, Programmcode und Darstellung%
}{%
}{%
  Das Urheberrecht schützt sowohl die persönlich geistige Schöpfung (Kreativität) von Personen als
  auch Datenbanken. Der Vortrag untersucht Fragen, die sich stellen, wenn virale Lizenzen (Copyleft)
  von Software, Darstellungsregeln (rendering rules)  und Datenbank (z.B. ODbL) im GIS oder Renderer
  zusammentreffen.%
}%


%%%%%%%%%%%%%%%%%%%%%%%%%%%%%%%%%%%%%%%%%%%

% time: Wednesday 15:30
% URL: https://pretalx.com/fossgis2020/talk/7BCWHD/

%

\noindent\abstractHSRundbau{%
  Thomas Baumann%
}{%
  QGIS im Produktivbetrieb%
}{%
Erfahrungsbericht zur Einführung von QGIS als\\ professionelles Planungswerkzeug%
}{%
  Seit Anfang 2016 habe ich bei einem mittelständischen Unternehmen QGIS als professionelles GIS mit
  Schwerpunkt Planung im Bereich des Breitbandausbaus eingeführt.
 In diesem Vortrag berichte ich über den Weg vom anfänglichen „fear, uncertainty, and
  doubt“ -Phänomen über erste Teilerfolge hin zu Akzeptanz von QGIS als etabliertes Werkzeug zur
  Unterstützung der Planer.
 Beleuchtet werden soll auch die technische Umsetzung zur Bereitstellung von QGIS im
  professionellen Umfeld.%
}%


%%%%%%%%%%%%%%%%%%%%%%%%%%%%%%%%%%%%%%%%%%%

% time: Wednesday 15:30
% URL: https://pretalx.com/fossgis2020/talk/S8HCFT/

%

\noindent\abstractWeismannhaus{%
  Michael Reichert%
}{%
  Umgang mit vorhandenen und fehlenden Relevanzinformationen in OpenStreetMap-Kartenstilen%
}{%
}{%
  Bei bestimmten Featureklassen ist auch in der automatisierten Kartographie die Trennung von
  wichtigen und weniger wichtigen Features von Bedeutung. OpenStreetMap-Daten enthalten Information
  über die Relevanz eines Objekts nur in bestimmten Fällen.
  Der Vortrag stellt vor, wie verschiedene OSM-Open-Source-Kartenstile vorhandenen
  Relevanzinformationen nutzen. Anschließend wird präsentiert, wie der
  OpenRailwayMap-Infra\-strukturstil Routenrelationen zur Bewertung von Haltestellen verwendet.%
}%


%%%%%%%%%%%%%%%%%%%%%%%%%%%%%%%%%%%%%%%%%%%

% time: Wednesday 16:00
% URL: https://pretalx.com/fossgis2020/talk/DH3H9D/

%
\newTimeslot{16:00}
\noindent\abstractHSAnatomie{%
  Bernhard Seeger%
}{%
  FAIRe Daten und FAIRe Software in der Biodiversitätsforschung%
}{%
}{%
  Der Verlust der biologischen Vielfalt und der Klimawandel gehören zu den zentralen
  Herausforderungen der Menschheit. Deshalb werden in der Wissenschaft vermehrt großflächig
  raumbezogene Vektor- und Rasterdaten auf Grundlage der FAIR-Prinzipien verwaltet und durch
  Anwendung skalierbarer Verarbeitungs- und Analysesoftware neue Erkenntnisse gewonnen. Dieser
  Beitrag gibt einen Überblick zu aktuellen Entwicklungen zum Forschungsdatenmanagement in
  Deutschland, Europa und darüber hinaus.%
}%


%%%%%%%%%%%%%%%%%%%%%%%%%%%%%%%%%%%%%%%%%%%

% time: Wednesday 16:00
% URL: https://pretalx.com/fossgis2020/talk/8WBR7B/

%

\noindent\abstractHSRundbau{%
  Günter Wagner%
}{%
  WebGIS kleiner Gemeinden mit QGIS-Server und Lizmap%
}{%
}{%
  Kleine Gemeinden (< 12.000 EW) haben auch Bedarf an WebGIS-Anwendungen, jedoch wenig finanzielle
  und personelle Ressourcen.
  Dieser Vortrag stellt die Realisierung solcher Anwendungen auf Basis von Open-Source-Komponenten
  vor. Dabei wird zuerst die Ausgangssituation dargestellt, dann die technische Realisierung
  erläutert und zum Abschluss werden Praxisbeispiele gezeigt.%
}%


%%%%%%%%%%%%%%%%%%%%%%%%%%%%%%%%%%%%%%%%%%%

% time: Wednesday 16:00
% URL: https://pretalx.com/fossgis2020/talk/PRCPBP/

%

\noindent\abstractWeismannhaus{%
  Mathias Gröbe%
}{%
  Reliefdarstellung mit Höhenlinien%
}{%
}{%
  Neben der Schummerung sind Höhenlinien die wichtigste kartographische Darstellungsmethode für das
  Relief in Karten. Anhand von Beispielen wird die Erstellung auf der Basis von digitalen
  Geländemodellen erklärt und wie man mit QGIS und GDAL ansprechende Höhenlinien erzeugen kann. Dazu
  wird noch erläutert wie die Abstände zwischen den Höhenlinien in Abhängigkeit vom Relief und dem
  Maßstab gewählt werden sollten.%
}%


%%%%%%%%%%%%%%%%%%%%%%%%%%%%%%%%%%%%%%%%%%%

% time: Wednesday 16:00
% URL: https://pretalx.com/fossgis2020/talk/LMZ3RU/

%

\noindent\abstractOther{%
  FOSSGIS e.V. Vorstand%
}{%
  AUF EIN WORT mit dem Vorstand%
}{%
}{%
  Wir im Vorstand wollen die Konferenz nutzen, um mit Euch, den Mitgliedern und der weiteren
  Community ins Gespräch zu kommen.%
}%
{%
  Meetingraum%
}%



%%%%%%%%%%%%%%%%%%%%%%%%%%%%%%%%%%%%%%%%%%%

% time: Wednesday 17:00
% URL: https://pretalx.com/fossgis2020/talk/YBRH7D/

%
\newTimeslot{17:00}
\noindent\abstractHSAnatomie{%
  Robert Klemm%
}{%
  Ein einheitlicher Frontend-Ansatz, um mehrere Routing-Lösungen im WebGIS zu nutzen.%
}{%
}{%
  Im Vortrag werden die weiterverbreiten Routing-Lösungen (OSRM, GraphHopper, PgRouting, ... )
  genutzt, um diese über ein einheitliches User-Interface einzubinden, Routing-Anfragen
  zu stellen und visuell in der Webkarte darzustellen.  Hierbei wird auf die unterschiedlichen
  Routing-Anforderungen eingegangen, die zusammenfassend die allgemeinen Anforderungen an das
  User-Interface stellen. Dieser Lösungsansatz wird am Bespiel vom OpenSource-WebGIS Mapbender
  durchgeführt.%
}%


%%%%%%%%%%%%%%%%%%%%%%%%%%%%%%%%%%%%%%%%%%%

% time: Wednesday 17:00
% URL: https://pretalx.com/fossgis2020/talk/FDAF37/

%

\noindent\abstractHSRundbau{%
  Astrid Emde%
}{%
  OSM-Daten in QGIS nutzen%
}{%
}{%
  Das Desktop-GIS QGIS bietet zahlreiche Möglichkeiten für den Umgang mit Vektor-, Rasterdaten und
  Diensten.
 Speziell für den Umgang mit OpenStreetMap-Daten existieren einige Möglichkeiten. Diese werden
  meist über Plugins bereit\-gestellt.
  Dieser Vortrag bietet einen Überblick über die Möglichkeiten der Nutzung von OpenStreetMap-Daten
  in QGIS.%
}%


%%%%%%%%%%%%%%%%%%%%%%%%%%%%%%%%%%%%%%%%%%%

% time: Wednesday 17:00
% URL: https://pretalx.com/fossgis2020/talk/L9NGAN/

%

\noindent\abstractWeismannhaus{%
  Andreas Weckbecker, Christine Dolezich%
}{%
  GeoPortal Koblenz~-- digital. vielschichtig. maßgebend.%
}{%
}{%
  Das Geoportal Koblenz wurde am 8. Novmber 2018 gelauncht. Nach einem Jahr Erfahrung und
  Verbesserung geben wir einen Abriß über die Funktionen und den weiteren Weg des
  OpenSource-Portals. Verwendnung findet das Geoportal sowohl innerhalb der Stadtverwaltung Koblenz
  als auch bei Internetnutzern.%
}%


%%%%%%%%%%%%%%%%%%%%%%%%%%%%%%%%%%%%%%%%%%%

% time: Wednesday 17:30
% URL: https://pretalx.com/fossgis2020/talk/339AY9/

%
\newTimeslot{17:30}
\noindent\abstractHSAnatomie{%
  Jakob Miksch%
}{%
  Einführung zu GDAL/OGR%
}{%
Geodaten mit der Kommandozeile verarbeiten%
}{%
  GDAL/OGR ist bekannt dafür alle erdenklichen Geodatenformate lesen und schreiben zu können.
 Es verfügt jedoch auch über zahlreiche Funktionen um Geo\-daten zu filtern, zu analysieren und zu
  verarbeiten. Dieser Vortrag demonstriert anhand von Beispielen wie man gängige Operationen für
  Raster- und Vektordaten mit Hilfe der Kommandozeile automatisert.%
}%


%%%%%%%%%%%%%%%%%%%%%%%%%%%%%%%%%%%%%%%%%%%

% time: Wednesday 17:30
% URL: https://pretalx.com/fossgis2020/talk/3WCSVT/

%

\noindent\abstractWeismannhaus{%
  Stefan Ziegler%
}{%
  Der ÖREB-Kataster%
}{%
Eine Ode an offene Standards und Software%
}{%
  Der Kataster der öffentlich-rechtlichen Eigentumsbeschränkungen (ÖREB-Kataster) ist das
  zuverlässige und offizielle Informationssystem für die wichtigsten öffentlich-rechtlichen
  Eigentumsbeschränkungen. Der Vortrag beschreibt wie mit offener Software, offenen Standards und
  der konsequenten Verwendung von gemeinsamen Datenmodellen der stark reglementierte ÖREB-Kataster
  effizient und transparent durch die katasterverantwortliche Stelle umgesetzt und eingeführt wurde.%
}%


%%%%%%%%%%%%%%%%%%%%%%%%%%%%%%%%%%%%%%%%%%%

% time: Wednesday 18:00
% URL: https://pretalx.com/fossgis2020/talk/TUAHKX/

%
\newTimeslot{18:00}
\noindent\abstractHSRundbau{%
  Claas Leiner%
}{%
  Schraffuren, die sich an der längsten Objektkante orientieren%
}{%
}{%
  Gebäude können sehr langgestreckt sein. Und da sie nicht immer in die gleiche Richtung verlaufen,
  sehen Gebäudeschraffuren oft merkwürdig aus. Hier wenige ganz lange Striche, dort ganz viele
  kurze.
  Mit einem kleinen QGIS-Modellertool lässt sich der Drehwinkel finden, der die Schraffur immer an
  der längsten Gebäudekante ausrichtet. Das geht natürlich auch für andere Objekte.%
}%


%%%%%%%%%%%%%%%%%%%%%%%%%%%%%%%%%%%%%%%%%%%

% time: Wednesday 18:00
% URL: https://pretalx.com/fossgis2020/talk/JCFGAX/

%

\noindent\abstractWeismannhaus{%
  Armin Retterath%
}{%
  GeoPortal.rlp unchained%
}{%
}{%
  Im August 2019 fand der 2. Relaunch des GeoPortal.rlp statt. Das System wurde hierzu komplett
  überarbeitet und viele bestehende Komponenten durch Django-Module ersetzt. Bei der Architektur des
  neuen Systems wurde vollständig auf die Einbettung von Portal\-komponenten in ein CMS verzichtet und
  die Redaktion erfolgt in einem angeschlossenen Mediawiki. Eine weitere Besonderheit ist der
  erstmalig realisierte direkte Zugriff auf die Geodatenkataloge Deutschlands und der EU über deren
  CSW-Interfaces.%
}%


%%%%%%%%%%%%%%%%%%%%%%%%%%%%%%%%%%%%%%%%%%%

% time: Wednesday 18:05
% URL: https://pretalx.com/fossgis2020/talk/VTQQJR/

%
\newTimeslot{18:05}
\noindent\abstractHSRundbau{%
  Marco Scheuble%
}{%
  NodeJS GeoServer und GeoWebCache Connectors%
}{%
}{%
  Mit den beiden Bibliotheken NodeJS-GeoServer-Connector und NodeJS-GeoWebCache-Connector hat die
  GEOINFO Applications AG zwei Bibliotheken entwickelt, die Anfang 2020 unter einer
  Open-Source-Lizenz veröffentlicht wurden. Die Node.js-Biblio\-theken ermöglichen die Administration
  des GeoServers und des GeoWebCaches durch einfache Funktionsaufrufe aus Node.js-basierten
  Applikationen heraus. Das aufwendige Erstellen und Parsen von XML-Dokumenten, sowie von
  HTTP-Requests und Responses fällt damit weg.%
}%


%%%%%%%%%%%%%%%%%%%%%%%%%%%%%%%%%%%%%%%%%%%

% time: Wednesday 18:10
% URL: https://pretalx.com/fossgis2020/talk/SE3BBN/

%
\newTimeslot{18:10}
\noindent\abstractHSRundbau{%
  Sebastian M. Ernst%
}{%
  Rethinking QGIS plugin packaging%
}{%
new opportunities through Anaconda?%
}{%
  QGIS plugin authors have rather limited options of defining plugin dependencies. Although
  inter-plugin-dependencies have been introduced in QGIS 3.6, the definition of arbitrary Python
  dependencies remains problematic. This is especially an issue on Windows, where plugin authors
  targeting non-technical users can only choose from Python packages shipped with OSGeo4W. Recent
  advances in packaging QGIS for the Anaconda Python distribution may offer a new path forward with
  unseen opportunities.%
}%


%%%%%%%%%%%%%%%%%%%%%%%%%%%%%%%%%%%%%%%%%%%

% time: Wednesday 18:15
% URL: https://pretalx.com/fossgis2020/talk/8D7WEC/

%
\newTimeslot{18:15}
\noindent\abstractHSRundbau{%
  Bernhard Ströbl%
}{%
  QGIS-Übersetzung%
}{%
}{%
  Stand der Übersetzung des Handbuchs und Aufruf zur Mitarbeit. Wir zeigen die Probleme auf und die
  Lösungen, die der QGIS-DE unternimmt.%
}%


%%%%%%%%%%%%%%%%%%%%%%%%%%%%%%%%%%%%%%%%%%%

% time: Wednesday 19:00
% URL: https://pretalx.com/fossgis2020/talk/K9TZSM/

%
\newTimeslot{19:00}
\noindent\abstractOther{%
%
}{%
  Schwätzli uffem Campus%
}{%
}{%
  Die Abendveranstaltung "`Schwätzli uffem Campus"' oder auf Hochdeutsch "`Der fachliche Austausch im
  Institutsviertel"' findet in der Mensa statt. Der Eintritt ist nur mit einem kostenlosen Ticket möglich. Wer es noch nicht bei der Anmeldung hinzugebucht hat kann es am Welcome Desk erhalten, sofern noch freie Plätze vorhanden sind.%
}%
{%
  Mensa%
}%



%%%%%%%%%%%%%%%%%%%%%%%%%%%%%%%%%%%%%%%%%%%
