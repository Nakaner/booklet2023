
% time: Wednesday 10:30
% URL: https://pretalx.com/fossgis2020/talk/JV7GRQ/

%
\newTimeslot{10:30}
\noindent\abstractHSAnatomie{%
  Wolfgang Hinsch%
}{%
  Die Welt als runde Sache auf der ebenen Karte%
}{%
}{%
  Es ist nicht möglich, die Erde verzerrungsfrei auf einer Karte darzustellen. Es wird gezeigt, mit
  welchem Abbildungssystem OSM arbeitet, welche Vor- und Nachteile es hat, und eine Auswahl
  relevanter alternativer Koordinatenbezugssysteme aufgezeigt. Dabei werden Begriffe wie
  Referenzellipsoid, Geoid, EPSG, WGS84, NHN etc. vorgestellt.%
}%


%%%%%%%%%%%%%%%%%%%%%%%%%%%%%%%%%%%%%%%%%%%

% time: Wednesday 11:00
% URL: https://pretalx.com/fossgis2020/talk/QMK3JN/

%
\newSmallTimeslot{11:00}
\noindent\abstractHSAnatomie{%
  Hanna Krüger%
}{%
  Ehrenamt im FOSSGIS e.V.%
}{%
}{%
  Was ist eigentlich der FOSSGIS e.V. und was ist sein Ziel, wie funktioniert er und wie sieht das
  Vereinsleben aus?%
}%

\sponsorBoxA{401-tib.png}{0.630\textwidth}{4}{%
\textbf{Medienpartner}\\
Die Technische Informationsbibliothek in Hannover ist die Deutsche
Zentrale Fachbibliothek für Technik und Naturwissenschaften. Sie versorgt in
ihren Spezialgebieten die nationale wie internationale Forschung und Industrie
mit Literatur und Information in gedruckter und elektronischer Form. Mit dem
AV-Portal bietet die TIB unter av.tib.eu eine Online-Plattform für
wissenschaftliche Videos.%
}


%%%%%%%%%%%%%%%%%%%%%%%%%%%%%%%%%%%%%%%%%%%

% time: Wednesday 11:25
% URL: https://pretalx.com/fossgis2020/talk/EZAY3D/

%
\newTimeslot{11:25}
\noindent\abstractHSAnatomie{%
  Marco Lechner%
}{%
  Was ist Open Source?%
}{%
}{%
  Der Vortrag stellt die Geschichte der Entwicklung von Open Source vor und geht auf wichtige
  Grundlagen ein.
  Ziel des FOSSGIS e.V. und der OSGeo ist die Förderung und Verbreitung freier Geographischer
  Informationssysteme (GIS) im Sinne Freier Software und Freier Geodaten. Dazu zählen auch
  Erstinformation und Klarstellung von typischen Fehlinformationen über Open Source und Freie
  Software, die sich über die Jahre festgesetzt haben.%
}%


%%%%%%%%%%%%%%%%%%%%%%%%%%%%%%%%%%%%%%%%%%%

% time: Wednesday 11:40
% URL: https://pretalx.com/fossgis2020/talk/UEWFWR/

%
\newSmallTimeslot{11:40}
\noindent\abstractHSAnatomie{%
  Thomas Skowron%
}{%
  Was ist OpenStreetMap?%
}{%
}{%
  Was ist OpenStreetMap? Die Enstehung, die heutige Bedeutung und die vielfältigen Nutzungsmöglichkeiten der offenen Weltkarte OpenStreetMap werden vorgestellt. Es werden Möglichkeiten zur Mitwirkung präsentiert.%
}%

%%%%%%%%%%%%%%%%%%%%%%%%%%%%%%%%%%%%%%%%%%%

% time: Wednesday 13:00
% URL: https://pretalx.com/fossgis2020/talk/TDNXQF/

%
\newSmallTimeslot{13:00}
\noindent\abstractHSRundbau{%
}{%
  Eröffnung%
}{%
}{%
  Feierliche Eröffnung der Konferenz durch Vertreter des FOSSGIS e.V. mit wichtigen Hinweisen
  zu Ablauf und Organisation.%
}%

\pagebreak
%%%%%%%%%%%%%%%%%%%%%%%%%%%%%%%%%%%%%%%%%%%

% time: Wednesday 13:45
% URL: https://pretalx.com/fossgis2020/talk/HEE3SU/

%
\newSmallTimeslot{13:45}
\noindent\abstractHSRundbau{%
  Guillaume Rischard%
}{%
  Pilgerstab in einer Hand, Brecheisen\\ in der anderen: wie man Geodaten öffnet%
}{%
 %wie man Geodaten öffnet%
}{%
In der EU geben die PSI-Richtlinien von 2003 und 2013 jedem das Recht auf
die Wiederverwendung von Daten aus dem öffentlichen Sektor. Oder tun sie
das wirklich? Was sagt das Informations\-weiter\-verwendungs\-gesetz und was
bedeutet es für uns? Was wird die Open-Data-Richtlinie von 2019, die bis
2021 umgesetzt sein muss, ermöglichen?
Sind solche Gesetze überhaupt notwendig? Im Urlaub im Kosovo tat Guillaume,
was jeder normale Mensch tun würde -- er begann zu kartographieren und nach\linebreak
offenen Geodaten zu suchen. Zusammen mit der örtlichen Community half er,
die Kosovo-Katasterbehörde davon zu überzeugen, zum ersten Mal Daten
freizugeben. Ohne Open-Data-Gesetze, aber mit Import-Mapathons und
Partnerschaften wurde OpenStreetMap zur besten Karte des Landes.
Zuvor war Guillaume der technische Leiter des luxemburgischen Projekts für
ein Open-Data-Portal. Als er begann, war Luxemburg Schlusslicht im
Open-Data-Index der EU. Es ist seitdem an erster Stelle und eines der
Länder mit der offensten Geodatenpolitik -- dank einer Kombination aus
Familien\-geschichte und OpenRailwayMap.
}%


\newTimeslot{15:00}


%%%%%%%%%%%%%%%%%%%%%%%%%%%%%%%%%%%%%%%%%%%

% time: Wednesday 15:00
% URL: https://pretalx.com/fossgis2020/talk/ZFQNNN/

%

\noindent\abstractHSRundbau{%
  Peter Heidelbach%
}{%
  Von ArcGis nach QGIS%
}{%
}{%
  Für die Konvertierung von ArcGIS-Projekten in QGIS-Projekte gibt es derzeit verschiedene Ansätze.
  Die australische Firma North Road entwicklet zur Zeit ein Tool zum Reverse-Engineering der
  Binärdateien. GeoCats Bridge priorisiert dagegen den Export der ArcGIS-Layer als Web-Services. An
  einer nativen Toolbox arbeitet die WhereGroup, welche Projektdaten als QGIS-XML exportiert. Der
  Vortrag beschreibt die unterschiedlichen Vorgehensweisen und stellt Vor- und Nachteile der einzelnen
  Verfahren gegenüber.%
}

%%%%%%%%%%%%%%%%%%%%%%%%%%%%%%%%%%%%%%%%%%%

% time: Wednesday 15:00
% URL: https://pretalx.com/fossgis2020/talk/GPMCKV/

%
\noindent\abstractHSAnatomie{%
  Pirmin Kalberer%
}{%
  2700 interaktive thematische Karten~--\linebreak
  Ein Fall für Vektortiles!%
}{%
}{%
  Die Webkarten des neuen Vogelatlas der Schweizer Vogelwarte bieten dank Vektortile-Technologie
  hohe Interaktivität bei geringem Resourcenbedarf. Der Vortrag zeigt die technischen Hintegründe,
  aber auch viele Karten!%
}

\sponsorBoxA{images-print/402-voc.pdf}{0.27\textwidth}{3}{%
  \textbf{Medienpartner C3VOC}\\
  \noindent Die Videoaufzeichnung dieser Konferenz erfolgt durch die Freiwilligen des
  CCC \mbox{Video} Operation Center.
}


%%%%%%%%%%%%%%%%%%%%%%%%%%%%%%%%%%%%%%%%%%%

% time: Wednesday 15:00
% URL: https://pretalx.com/fossgis2020/talk/YQGPNJ/

%

\noindent\abstractWeismannhaus{%
  Christoph Hormann%
}{%
  Weniger ist mehr~-- zur Auswahl darzustellender Elemente in der digitalen \mbox{Kartographie}%
}{%
}{%
  Die Auswahl von dem, was man in einer Karte darstellt, ist von entscheidender Bedeutung für deren
  Lesbarkeit.  In Karten auf Grundlage von OpenStreetMap-Daten wird diese Auswahl für viele wichtige
  Elemente durch subjektive Klassifizierungen oder die Größe subjektiver Beschriftungs-Geometrien
  vorgenommen, welche eigentlich in OpenStreetMap nichts zu tun haben.  Dieser Vortrag gibt einen
  Überblick über Ansätze zur Bewertung der Bedeutung von Objekten für den Zweck der Auswahl bei der
  Darstellung.%
}%
\vspace{1.92\baselineskip}
\noindent\sponsorBoxA{207-mapwebbing}{0.42\textwidth}{3}{%
\textbf{Bronzesponsor}\\
mapwebbing, gegr. 2008 von Lars Lingner, Dipl.-Inf. (FH), spezialisiert auf
die Planung und Erstellung individueller Kartenanwendungen. Kernkompetenz
ist die Entwicklung maßgeschneiderter Konzepte zur Verarbeitung von
Geodaten sowie die Aufbereitung von Kartenmaterial für den professionellen
Druck.%
}


\newTimeslot{15:30}

%%%%%%%%%%%%%%%%%%%%%%%%%%%%%%%%%%%%%%%%%%%

% time: Wednesday 15:30
% URL: https://pretalx.com/fossgis2020/talk/7BCWHD/

%

\noindent\abstractHSRundbau{%
  Thomas Baumann%
}{%
  QGIS im Produktivbetrieb%
}{%
Erfahrungsbericht zur Einführung von QGIS als\\ professionelles Planungswerkzeug%
}{%
  Seit Anfang 2016 habe ich bei einem mittelständischen Unternehmen QGIS als professionelles GIS mit
  Schwerpunkt Planung im Bereich des Breitbandausbaus eingeführt.
 In diesem Vortrag berichte ich über den Weg vom anfänglichen „fear, uncertainty, and
  doubt“ -Phänomen über erste Teilerfolge hin zu Akzeptanz von QGIS als etabliertes Werkzeug zur
  Unterstützung der Planer.
 Beleuchtet werden soll auch die technische Umsetzung zur Bereitstellung von QGIS im
  professionellen Umfeld.%
}%

%%%%%%%%%%%%%%%%%%%%%%%%%%%%%%%%%%%%%%%%%%%

% time: Wednesday 15:30
% URL: https://pretalx.com/fossgis2020/talk/MYTFXF/

%
\noindent\abstractHSAnatomie{%
  Falk Zscheile%
}{%
  Kartenherstellung zwischen Lizenzen, Daten, Programmcode und Darstellung%
}{%
}{%
  Das Urheberrecht schützt sowohl die persönlich geistige Schöpfung (Kreativität) von Personen als
  auch Datenbanken. Der Vortrag untersucht Fragen, die sich stellen, wenn virale Lizenzen (Copyleft)
  von Software, Darstellungsregeln (rendering rules)  und Datenbank (z.\,B. ODbL) im GIS oder Renderer
  zusammentreffen.%
}%


%%%%%%%%%%%%%%%%%%%%%%%%%%%%%%%%%%%%%%%%%%%

% time: Wednesday 15:30
% URL: https://pretalx.com/fossgis2020/talk/S8HCFT/

%

\noindent\abstractWeismannhaus{%
  Michael Reichert%
}{%
  Umgang mit vorhandenen und \mbox{fehlenden} Relevanzinformationen in OpenStreetMap-Kartenstilen%
}{%
}{%
  Bei bestimmten Featureklassen ist auch in der automatisierten Kartographie die Trennung von
  wichtigen und weniger wichtigen Features von Bedeutung. OpenStreetMap-Daten enthalten Information
  über die Relevanz eines Objekts nur in bestimmten Fällen.
  Der Vortrag stellt vor, wie verschiedene OSM-Open-Source-Kartenstile vorhandenen
  Relevanzinformationen nutzen. Anschließend wird präsentiert, wie der
  OpenRailwayMap-Infra\-strukturstil Routenrelationen zur Bewertung von Haltestellen verwendet.%
}%
\sponsorBoxA{301-vag}{0.52\textwidth}{3}{%
  \textbf{Mobilitäts-Sponsor}\\
  Die Freiburger Verkehrs AG ist die Anbieterin der vernetzten und
  klimafreundlichen Mobilität in Freiburg. Rund 80,5 Mio. Fahrgäste im Jahr
  nutzen ihre Stadtbahnen, Busse und Leihräder. So leistet die VAG einen
  wichtigen Beitrag zur Verbesserung der Mobilität in Freiburg~-- auch mit der
  App VAG mobil, mit der die Kombination von verschiedenen Mobilitätsangeboten
  noch einfacher und schneller werden soll. www.vag-freiburg.de
}


\newTimeslot{16:00}


%%%%%%%%%%%%%%%%%%%%%%%%%%%%%%%%%%%%%%%%%%%

% time: Wednesday 16:00
% URL: https://pretalx.com/fossgis2020/talk/8WBR7B/

%

\noindent\abstractHSRundbau{%
  Günter Wagner%
}{%
  Web-GIS kleiner Gemeinden mit QGIS-Server und Lizmap%
}{%
}{%
  Kleine Gemeinden (< 12\,000 Einwohner) haben auch Bedarf an Web-GIS-Anwendungen, jedoch wenig finanzielle
  und personelle Ressourcen.
  Dieser Vortrag stellt die Realisierung solcher Anwendungen auf Basis von Open-Source-Komponenten
  vor. Dabei wird zuerst die Ausgangssituation dargestellt, dann die technische Realisierung
  erläutert und zum Abschluss werden Praxisbeispiele gezeigt.%
}%

%%%%%%%%%%%%%%%%%%%%%%%%%%%%%%%%%%%%%%%%%%%

% time: Wednesday 16:00
% URL: https://pretalx.com/fossgis2020/talk/DH3H9D/

%
\noindent\abstractHSAnatomie{%
  Bernhard Seeger%
}{%
  FAIRe Daten und FAIRe Software in der \mbox{Biodiversitätsforschung}%
}{%
}{%
  Der Verlust der biologischen Vielfalt und der Klimawandel gehören zu den zentralen
  Herausforderungen der Menschheit. Deshalb werden in der Wissenschaft vermehrt großflächig
  raumbezogene Vektor- und Rasterdaten auf Grundlage der FAIR-Prinzipien verwaltet und durch
  Anwendung skalierbarer Verarbeitungs- und Analysesoftware neue Erkenntnisse gewonnen. Dieser
  Beitrag gibt einen Überblick zu aktuellen Entwicklungen zum Forschungsdatenmanagement in
  Deutschland, Europa und darüber hinaus.%
}%

\newpage

%%%%%%%%%%%%%%%%%%%%%%%%%%%%%%%%%%%%%%%%%%%

% time: Wednesday 16:00
% URL: https://pretalx.com/fossgis2020/talk/PRCPBP/

%

\noindent\abstractWeismannhaus{%
  Mathias Gröbe%
}{%
  Reliefdarstellung mit Höhenlinien%
}{%
}{%
  Neben der Schummerung sind Höhenlinien die wichtigste kartographische Darstellungsmethode für das
  Relief in Karten. Anhand von Beispielen wird die Erstellung auf der Basis von digitalen
  Geländemodellen erklärt und wie man mit QGIS und GDAL ansprechende Höhenlinien erzeugen kann. Dazu
  wird noch erläutert, wie die Abstände zwischen den Höhenlinien in Abhängigkeit vom Relief und dem
  Maßstab gewählt werden sollten.%
}%


%%%%%%%%%%%%%%%%%%%%%%%%%%%%%%%%%%%%%%%%%%%

% time: Wednesday 16:00
% URL: https://pretalx.com/fossgis2020/talk/LMZ3RU/

%

\noindent\abstractOther{%
  Vorstand%
}{%
  Auf ein Wort mit dem Vorstand%
}{%
}{%
  Wir im Vorstand wollen die Konferenz nutzen, um mit Euch, den Mitgliedern und der weiteren
  Community ins Gespräch zu kommen.%
}%
{%
  R\,00\,004%
}%

\sponsorBoxA{403-osgeolive.pdf}{0.55\textwidth}{3}{%
\textbf{Medienpartner}\\
OSGeoLive bietet Ihnen die Möglichkeit, freie und Open"=Source"=Software in Verbindung mit Geodaten
auszuprobieren, ohne aufwendige Installationen durchführen zu müssen. Das OSGeoLive"=Projekt wird von
der OSGeo Foundation und vielen weiteren Aktiven getragen.%
}





\newTimeslot{17:00}
%%%%%%%%%%%%%%%%%%%%%%%%%%%%%%%%%%%%%%%%%%%
% time: Wednesday 17:00
% URL: https://pretalx.com/fossgis2020/talk/FDAF37/
\noindent\abstractHSRundbau{%
  Astrid Emde%
}{%
  OSM-Daten in QGIS nutzen%
}{%
}{%
  Das Desktop-GIS QGIS bietet zahlreiche Möglichkeiten für den Umgang mit Vektor-, Rasterdaten und
  Diensten.
 Speziell für den Umgang mit OpenStreetMap-Daten existieren einige Möglichkeiten. Diese werden
  meist über Plugins bereit\-gestellt.
  Dieser Vortrag bietet einen Überblick über die Möglichkeiten der Nutzung von OpenStreetMap-Daten
  in QGIS.%
}%

%%%%%%%%%%%%%%%%%%%%%%%%%%%%%%%%%%%%%%%%%%%
% time: Wednesday 17:00
% URL: https://pretalx.com/fossgis2020/talk/YBRH7D/
\noindent\abstractHSAnatomie{%
  Robert Klemm%
}{%
  Ein einheitlicher Frontend-Ansatz, um mehrere Routing-Lösungen im Web-GIS zu nutzen%
}{%
}{%
  Im Vortrag werden die weiterverbreiten Routing-Lösungen (OSRM, GraphHopper, PgRouting, \dots)
  genutzt, um diese über ein einheitliches User-Interface einzubinden, Routing-Anfragen
  zu stellen und visuell in der Webkarte darzustellen.  Hierbei wird auf die unterschiedlichen
  Routing-Anforderungen eingegangen, die zusammenfassend die allgemeinen Anforderungen an das
  User-Interface stellen. Dieser Lösungsansatz wird am Bespiel vom Open-Source-Web-GIS Mapbender
  durchgeführt.%
}%

\newpage

%%%%%%%%%%%%%%%%%%%%%%%%%%%%%%%%%%%%%%%%%%%

% time: Wednesday 17:00
% URL: https://pretalx.com/fossgis2020/talk/L9NGAN/

%

\noindent\abstractWeismannhaus{%
  Andreas Weckbecker, Christine Dolezich%
}{%
  GeoPortal Koblenz~-- digital,\linebreak
  vielschichtig, maßgebend%
}{%
}{%
  Das Geoportal Koblenz wurde am 8. Novmber 2018 gelauncht. Nach einem Jahr Erfahrung und
  Verbesserung geben wir einen Abriss über die Funktionen und den weiteren Weg des
  Open-Source-Portals. Verwendnung findet das Geoportal sowohl innerhalb der Stadtverwaltung Koblenz
  als auch bei Internetnutzern.%
}%
\sponsorBoxA{209-intevation}{0.42\textwidth}{4}{%
  \textbf{Bronzesponsor}\\
  Die Intevation GmbH ist seit 20 Jahren ein unabhängiger IT"=Dienstleister
  aus Osnabrück mit 25 Mitarbeitenden. Ob Planung, Umsetzung oder
  Weiterentwicklung von bestehenden und neuen Software-Lösungen~-- wir
  setzen konsequent auf Freie Software (Open Source). Damit Sie
  unabhängig bleiben!
}


\sponsorBoxA{405-frelo}{0.28\textwidth}{3}{%
\textbf{Mobilitätspartner}\\
Frelo ist der Bikesharing-Anbieter in Freiburg und Teil von Nextbike.  Als
Teilnehmer der Konferenz haben Sie bei der Registrierung einen Gutschein
zur Nutzung der Räder in Freiburg während der Konferenz erhalten.
Nutzungsinstruktionen hängen am Welcome Desk aus bzw. sind dort
erhältlich. Weitere Informationen zu Frelo finden Sie auf
https://www.frelo-freiburg.de/de/
}


%%%%%%%%%%%%%%%%%%%%%%%%%%%%%%%%%%%%%%%%%%%

% time: Wednesday 17:30
% URL: https://pretalx.com/fossgis2020/talk/339AY9/

%
\newTimeslot{17:30}
\noindent\abstractHSAnatomie{%
  Jakob Miksch%
}{%
  \emph{Demo-Session:} Einführung zu GDAL/OGR%
}{%
Geodaten mit der Kommandozeile verarbeiten%
}{%
  GDAL/OGR ist bekannt dafür alle erdenklichen Geodatenformate lesen und schreiben zu können.
 Es verfügt jedoch auch über zahlreiche Funktionen um Geo\-daten zu filtern, zu analysieren und zu
  verarbeiten. Dieser Vortrag demonstriert anhand von Beispielen wie man gängige Operationen für
  Raster- und Vektordaten mit Hilfe der Kommandozeile automatisert.%
}%


%%%%%%%%%%%%%%%%%%%%%%%%%%%%%%%%%%%%%%%%%%%

% time: Wednesday 17:30
% URL: https://pretalx.com/fossgis2020/talk/3WCSVT/

%

\noindent\abstractWeismannhaus{%
  Stefan Ziegler%
}{%
  Der ÖREB-Kataster%
}{%
Eine Ode an offene Standards und Software%
}{%
  Der Kataster der öffentlich-rechtlichen Eigentumsbeschränkungen (ÖREB-Kataster) ist das
  zuverlässige und offizielle Informationssystem für die wichtigsten öffentlich-rechtlichen
  Eigentumsbeschränkungen. Der Vortrag beschreibt wie mit offener Software, offenen Standards und
  der konsequenten Verwendung von gemeinsamen Datenmodellen der stark reglementierte ÖREB-Kataster
  effizient und transparent durch die katasterverantwortliche Stelle umgesetzt und eingeführt wurde.%
}%

\enlargethispage{1\baselineskip}
\newSmallTimeslot{18:00}
\noindent\abstractHSRundbau{%
}{%
  Lightning Talks%
}{%
}{%
  \vspace{-2em}
  \begin{itemize}
    \RaggedRight
    \setlength{\itemsep}{-2pt} % Aufzählungspunktabstand auf 0
  \item \emph{Claas Leiner:} Schraffuren, die sich an der längsten Objektkante orientieren
    \item \emph{Marco Scheuble:} NodeJS-Geoserver- und -GeoWebCache-Connectors
    \item \emph{Sebastian M. Ernst:} Rethinking QGIS plugin packaging – new opportunities through Anaconda?
    \item \emph{Bernhard Ströbl:} QGIS-Übersetzung
  \end{itemize}
  \justifying
}

%%%%%%%%%%%%%%%%%%%%%%%%%%%%%%%%%%%%%%%%%%%
% time: Wednesday 18:00
% URL: https://pretalx.com/fossgis2020/talk/JCFGAX/
\noindent\abstractWeismannhaus{%
  Armin Retterath%
}{%
  GeoPortal.rlp unchained%
}{%
}{%
  Im August 2019 fand der 2. Relaunch des GeoPortal.rlp statt. Das System wurde hierzu komplett
  überarbeitet und viele bestehende Komponenten durch Django-Module ersetzt. Bei der Architektur des
  neuen Systems wurde vollständig auf die Einbettung von Portal\-komponenten in ein CMS verzichtet und
  die Redaktion erfolgt in einem angeschlossenen Mediawiki. Eine weitere Besonderheit ist der
  erstmalig realisierte direkte Zugriff auf die Geodatenkataloge Deutschlands und der EU über deren
  CSW-Interfaces.%
}%


%%%%%%%%%%%%%%%%%%%%%%%%%%%%%%%%%%%%%%%%%%%

% time: Wednesday 19:00
% URL: https://pretalx.com/fossgis2020/talk/K9TZSM/

%
\newSmallTimeslot{19:00}
\noindent\abstractOther{%
%
}{%
  Schwätzli uffem Campus\label{schwaetzli}%
}{%
}{%
  Die Abendveranstaltung \emph{Schwätzli uffem Campus} oder auf Hochdeutsch \emph{Der fachliche Austausch im
    Institutsviertel} findet in der Mensa statt. Der Eintritt ist nur mit einem gebuchten Ticket möglich. Wer es noch nicht bei der Anmeldung hinzugebucht hat kann es am Welcome Desk erhalten, sofern noch freie Plätze vorhanden sind.%
}%
{%
  Mensa%
}%



%%%%%%%%%%%%%%%%%%%%%%%%%%%%%%%%%%%%%%%%%%%
