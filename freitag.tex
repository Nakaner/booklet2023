
% time: Friday 08:00
% URL: https://pretalx.com/fossgis2020/talk/NBTG8B/
\renewcommand{\conferenceDay}{\freitag}

%

%%%%%%%%%%%%%%%%%%%%%%%%%%%%%%%%%%%%%%%%%%%

% time: Friday 09:00
% URL: https://pretalx.com/fossgis2020/talk/TQYRU8/

%
\newTimeslot{09:00}
\noindent\abstractHSAnatomie{%
  Holger Bruch%
}{%
  Mobil in Herrenberg mit Digitransit\\ und offenen Daten%
}{%
}{%
  Wie lassen sich Menschen zum Umstieg auf den Umweltverbund bewegen?
 Mit "`Mobil-in-Herrenberg"' setzt die Stadt Herrenberg auf eine intermodale Auskunft auf Basis
  offener Software \& Daten, die Reisenden Auskunft über Alternativen zur Fahrt mit dem PKW aufzeigen
  sollen.%
}%


%%%%%%%%%%%%%%%%%%%%%%%%%%%%%%%%%%%%%%%%%%%

% time: Friday 09:00
% URL: https://pretalx.com/fossgis2020/talk/HZ8XES/

%

\noindent\abstractHSRundbau{%
  Felix Kunde%
}{%
  Hochverfügbare PostGIS-Cluster\\ auf Kubernetes%
}{%
}{%
  Container-basierte IT-Infrastrukturen sind bereits seit einigen Jahren in Mode und werden im Zuge
  von Cloud-basierten\linebreak Irgendwas-as-a-service Anwendungen immer mehr zum Standard. Im Vortrag wird
  vorgestellt, wie sich ein hochverfügbarer PostGIS-Cluster auf Kubernetes betreiben lässt.%
}%


%%%%%%%%%%%%%%%%%%%%%%%%%%%%%%%%%%%%%%%%%%%

% time: Friday 09:00
% URL: https://pretalx.com/fossgis2020/talk/AP9CC8/

%

\noindent\abstractWeismannhaus{%
  Karsten Prehn%
}{%
  OBM completeness~-- Ein OpenStreetMap-Tool zur Bewertung der Vollständigkeit des OSM-Gebäudebestandes%
}{%
}{%
  \emph{OBM (OpenBuildingMap) completeness} ist eine App zur Bestimmung der Vollständigkeit
  (completeness) des OpenStreetMap- (OSM)-Gebäudebestandes, die dem Gedanken des croud-sourcing
  folgt. Zur Bemessung der Vollständigkeit müssen OSM-Gebäude\-grundflächen mit dem
  Realwelt-Gebäudebestand auf Satellitenbildern innerhalb definierter Kacheln verglichen werden und
  der Grad der Vollständigkeit innerhalb der jeweiligen Kachel anhand der Stufen inkomplett, fast
  komplett oder komplett festgestellt werden.%
}%


%%%%%%%%%%%%%%%%%%%%%%%%%%%%%%%%%%%%%%%%%%%

% time: Friday 09:30
% URL: https://pretalx.com/fossgis2020/talk/3F8JCB/

%
\newTimeslot{09:30}
\noindent\abstractHSAnatomie{%
  Dr Julia Richter, Thomas Graichen%
}{%
  OPENER: Offene Plattform für die Crowd-basierte Erfassung von Informationen zu\\ Barrieren an Haltestellen im ÖPNV%
}{%
}{%
  Im Projekt OPENER haben wir eine Open Source Applikation zur Erfassung von Informationen zu
  Barrieren an Haltestellen im ÖPNV entwickelt, welche eine Crowd-basierte flächendeckende und
  lückenlose Erfassung ermöglicht. Die aktuell in einer separaten Datenbank erfassten Daten sollen
  ins OSM zurückgespeist werden, wobei der Beitrag die Applikation präsentiert sowie rechtliche und
  technische Aspekte zur Diskussion stellt.%
}%


%%%%%%%%%%%%%%%%%%%%%%%%%%%%%%%%%%%%%%%%%%%

% time: Friday 09:30
% URL: https://pretalx.com/fossgis2020/talk/7MSDDM/

%

\noindent\abstractHSRundbau{%
  Richard Bischof%
}{%
  Skalierbare Plattform zur Verarbeitung von Geodaten auf Basis von Kubernetes%
}{%
}{%
  Die Digitalisierung fordert die Verfügbarkeit jeglicher staatlichen Dienstleistung über digitale
  Kanäle. Geodatenhaltende Stellen müssen dazu eine vergleichsweise große Datenmenge verarbeiten.
  Monolithische Altverfahren sind aufgrund mangelnder Zuverlässigkeit, Flexibilität und
  Skalierbarkeit nahezu ungeeignet, diesen Anforderungen adäquat zu begegnen. Vorgestellt wird, wie
  durch moderne Architekturen und Technologien des Cloud Computing Geodaten effizient verarbeitet
  werden können.%
}%


%%%%%%%%%%%%%%%%%%%%%%%%%%%%%%%%%%%%%%%%%%%

% time: Friday 09:30
% URL: https://pretalx.com/fossgis2020/talk/XKUMED/

%

\noindent\abstractWeismannhaus{%
  Thomas Beutin%
}{%
  Asynchrones Python-basiertes Taskset zur permanenten Aktualisierung einer lokalen weltweiten OSM-basierten Gebäudedatenbank%
}{%
}{%
  Wir stellen ein Python-basiertes Taskset zur permanenten Aktualisierung einer weltweiten, sich ständig aktualisierenden OSM-basierten Datenbank von Gefährdungsindikatoren für Gebäude vor.%
}%


%%%%%%%%%%%%%%%%%%%%%%%%%%%%%%%%%%%%%%%%%%%

% time: Friday 10:00
% URL: https://pretalx.com/fossgis2020/talk/UDVAAL/

%
\newTimeslot{10:00}
\noindent\abstractHSAnatomie{%
  Dr. Dirk Schlierkamp-Voosen%
}{%
  Digitale Bahnhofspläne für die Reisenden\-information der Deutschen Bahn%
}{%
}{%
  Auf Basis digitaler Bahnhofspläne wird ein Kartendienst mit Fußgängerrouting bereitgestellt, der
  die Orientierung und die Navigation im Bahnhof erleichtert.%
}%


%%%%%%%%%%%%%%%%%%%%%%%%%%%%%%%%%%%%%%%%%%%

% time: Friday 10:00
% URL: https://pretalx.com/fossgis2020/talk/K7VNXP/

%

\noindent\abstractHSRundbau{%
  Michel Peltriaux%
}{%
  Mr. Map~-- Open Source Service Registry%
}{%
}{%
  Vorstellung einer Open Source Service Registry für Geodatendienste, welche sich zur Zeit in der
  Entwicklung durch die GDI-RP befindet. Das System schließt an alte Tugenden bestehender
  Geodateninfrastrukturen an und bietet darüber hinaus Funktionen, welche den Anforderungen von
  modernen Webapplikationen entsprechen.%
}%


%%%%%%%%%%%%%%%%%%%%%%%%%%%%%%%%%%%%%%%%%%%
% Lightning-Talks
%%%%%%%%%%%%%%%%%%%%%%%%%%%%%%%%%%%%%%%%%%%

% time: Friday 11:00
% URL: https://pretalx.com/fossgis2020/talk/7EPGZD/

%
\newTimeslot{11:00}
\noindent\abstractHSAnatomie{%
  Pirmin Kalberer%
}{%
  Vektor-Tiles Karten erstellen mit t-rex%
}{%
}{%
  Vektortiles haben das Potential, die bewährten Rasterkarten in vielen Bereichen abzulösen oder
  mindestens massgeblich zu ergänzen. Dieser Vortrag zeigt, wie Vektortiles generiert werden können
  und was dabei zu beachten ist. Weitere Themen sind das Styling der Karten mit Mapbox GL JSON und
  die Publikation mit OpenLayers 6 und Mapbox GL JS.%
}%


%%%%%%%%%%%%%%%%%%%%%%%%%%%%%%%%%%%%%%%%%%%

% time: Friday 11:00
% URL: https://pretalx.com/fossgis2020/talk/YFEKKC/

%

\noindent\abstractHSRundbau{%
  Marco Lechner%
}{%
  Open Source GIS-Komponenten im radiologischen Notfall-Informationssystem des Bundes%
}{%
}{%
  Das radiologische Notfall-Informationssystem des Bundes (IMIS3) hat Ende 2019 das vorhergehende
  proprietäre System abgelöst. IMIS3 wurde dabei konsequent aus freien Komponenten aufgebaut und um
  räumliche Funktionen erweitert. Aufbau und Zusammenspiel der Komponenten des IMIS3 sowie die
  Umsetzung der mit der Entwicklung einhergehenden Open-Source-Strategie in einer Behörde des Bundes
  werden ebenso dargestellt wie der Impact der Entwicklungen auf bestehende Projekte aus dem
  Umfeld des FOSSGIS.%
}%


%%%%%%%%%%%%%%%%%%%%%%%%%%%%%%%%%%%%%%%%%%%

% time: Friday 11:00
% URL: https://pretalx.com/fossgis2020/talk/PWE3PX/

%

\noindent\abstractWeismannhaus{%
  Robert Danziger%
}{%
  Mannheimer Mapathons%
}{%
Integration fördern, humanitäre Hilfe leisten%
}{%
  Das Project \emph{MAnnheimer MAPAthons ``MAMAPA''} (https://mamapa.org) organisiert seit Anfang 2018
  Mapathons   unter gemeinsamer Beteiligung von neu Zugewanderten und "`Einheimischen"'. Durch das
  Kartografieren im Tandem wird über die humanitäre Hilfe hinaus ein konkreter Beitrag zur
  Integration geleistet. Das Projekt wird sowohl von lokalen Integrationsträgern und Behörden  als
  auch von \emph{CartONG} (https://cartong.org), France und Mitgliedern des Geografischen Instituts der
  Uni Heidelberg unterstützt.%
}%


%%%%%%%%%%%%%%%%%%%%%%%%%%%%%%%%%%%%%%%%%%%

% time: Friday 11:30
% URL: https://pretalx.com/fossgis2020/talk/V8ZBMG/

%
\newTimeslot{11:30}
\noindent\abstractHSRundbau{%
  Sandro Mani%
}{%
  KADAS Albireo: Ein vereinfachtes QGIS für jedermann%
}{%
}{%
  KADAS Albireo ist eine Desktop-GIS-Anwendung, die für die Schweizer Armee zum Einsatz auf mehreren
  tausend Arbeitsplätzen entwickelt entwickelt wurde. Basierend auf den QGIS-Bibliotheken, bietet
  sie eine stark vereinfachte Benutzeroberfläche an sowie zahlreiche speziell entwickelte Funktionen
  an. Sie ein Beispiel einer vollständig personalisierten QGIS Anwendung, die die leistungsstarke
  und vielseitige QGIS-API nutzt und gleichzeitig einen separaten Entwicklungszyklus ermöglicht.%
}%


%%%%%%%%%%%%%%%%%%%%%%%%%%%%%%%%%%%%%%%%%%%

% time: Friday 11:30
% URL: https://pretalx.com/fossgis2020/talk/WC7SVB/

%

\noindent\abstractWeismannhaus{%
  Christopher Lorenz, Lars Lingner%
}{%
  Community Arbeit~-- Ein Einblick in die Berliner OSM/FOSSGIS-Community%
}{%
}{%
  Der Vortrag zeigt am Beispiel der Berliner Community welche Aktivitäten neben der Online-Welt von
  OpenStreetMap möglich sind, welche Hürden dabei überwunden werden müssen und man nie aufgeben
  sollte wenn etwas nicht vom Anfang an so läuft wie erwartet. Ein Einblick in die Arbeit von zwei
  engagierten OSM- und FOSSGIS-Mitgliedern.%
}%


%%%%%%%%%%%%%%%%%%%%%%%%%%%%%%%%%%%%%%%%%%%

% time: Friday 12:00
% URL: https://pretalx.com/fossgis2020/talk/MF7FZD/

%
\newTimeslot{12:00}
\noindent\abstractHSRundbau{%
  Gunnar Ströer%
}{%
  WPS für kommunale GDIs~-- Eine Fallstudie über den Mehrwert von Web Processing Services (WPS) am Beispiel der Geodateninfrastruktur Freiburg (GDI-FR)%
}{%
}{%
  Die zunehmende Digitalisierung der Verwaltungen schafft den Bedarf an Automatisierung komplexer
  Prozesse über ein breites Spektrum an Disziplinen. Solche Prozesse verwenden oft Geodaten, was
  eine GDI zum idealen Ausgangspunkt macht. Der Vortrag beruht auf einer Studie zur Untersuchung von
  WPS im kommunalen Umfeld. Ein Anwendungsfall umfasst die Evakuierungsplanung bei der
  Kampfmittelbeseitigung und demonstriert die Anwendbarkeit einer aus acht Prozessen bestehenden
  Prozesskette.%
}%


%%%%%%%%%%%%%%%%%%%%%%%%%%%%%%%%%%%%%%%%%%%

% time: Friday 12:00
% URL: https://pretalx.com/fossgis2020/talk/XBCSPZ/

%

\noindent\abstractWeismannhaus{%
  Joachim Kast%
}{%
  Rettungspunkte~-- Im Prinzip ganz einfach, aber ...%
}{%
}{%
  Rettungspunkte sind definierte Orte (meistens) im Wald,  an denen in fast allen Bundesländern
  Schilder mit einer Referenznummer angebracht sind. Diese Standorte werden in OSM seit mindestens
  2008 als highway=emergency\_access\_point erfasst. Seit 2014 werden die amtlichen Daten der meisten
  Bundesländer unter CC-BY-ND veröffentlicht, wodurch nun Qualitätskontrollen möglich sind. In dem
  Vortrag werden die verwendeten Methoden und die teilweise überraschenden Ergebnisse vorgestellt.%
}%


%%%%%%%%%%%%%%%%%%%%%%%%%%%%%%%%%%%%%%%%%%%

% time: Friday 13:30
% URL: https://pretalx.com/fossgis2020/talk/EVHGAH/

%
\newTimeslot{13:30}
\noindent\abstractHSRundbau{%
  Roman Härdi%
}{%
  Infrastruktur-Mappen mit dem (e-)Bike%
}{%
}{%
  Was sind die Vorteile des (e-)Bike-Mapping? Wie kann man Objekte mit der Kamera am besten und
  schnellsten (während der Fahrt) erfassen und präzise georeferenzieren

  Ich zeige wie ich das Mappen mit GPS, Fotoapparat und e-Bike optimiert habe und welche Vorteile
  diese Art des Mappen speziell für Infrastruktur (z.B. Gas-Pipelines) bietet an einem Beispiel.%
}%


%%%%%%%%%%%%%%%%%%%%%%%%%%%%%%%%%%%%%%%%%%%

% time: Friday 14:00
% URL: https://pretalx.com/fossgis2020/talk/JYFVQL/

%
\newTimeslot{14:00}
\noindent\abstractHSRundbau{%
  Danijel Schorlemmer%
}{%
  Erdbeben und OpenStreetMap%
}{%
}{%
  Zur Berechnung der möglichen Folgen von Erdbeben, ist die Kenntnis der Lage, Grösse und Typ von
  Gebäuden, ihr Wiederbeschaffungswert und die Anzahl der Bewohner nach Tageszeit notwendig.
  Mithilfe von OpenStreetMap und weiteren offenen Daten erzeugen wir ein globales, dynamisches,
  algorithmisches, und reproduzierbares Expositionsmodell zur probabilistischen Beschreibung dieser
  Parameter für jedes Gebäude der Welt, das mit OpenStreetMap wächst und sich verändert.%
}%


%%%%%%%%%%%%%%%%%%%%%%%%%%%%%%%%%%%%%%%%%%%

% time: Friday 15:00
% URL: https://pretalx.com/fossgis2020/talk/HYQSRM/

%
\newTimeslot{15:00}
\noindent\abstractHSRundbau{%
  FOSSGIS Konferenz%
}{%
  Abschlussveranstaltung und Sektempfang%
}{%
}{%
  Nemo voluptas ea mollitia consequatur vel. Error nulla nulla adipisci nisi distinctio maiores enim
  rem. Nam quo dolor tenetur. Rem quia consequatur voluptas. Temporibus nam nisi architecto
  perferendis saepe velit ipsam.%
}%


%%%%%%%%%%%%%%%%%%%%%%%%%%%%%%%%%%%%%%%%%%%

% time: Friday 15:30
% URL: https://pretalx.com/fossgis2020/talk/LKNHLJ/

%
\newTimeslot{15:30}
\noindent\abstractHSRundbau{%
  Johannes Kröger%
}{%
  FOSSGIS-Jeopardy%
}{%
}{%
  Das FOSSGIS-Jeopardy bietet als Fortsetzung der OSM-Quizze der Vorjahre wieder spannende Fragen zu
  (mehr oder weniger) wissenswerten Fakten und vor allem viel Spaß für jung und alt, alt und neu


  Du kennst sämtliche Parameter der OpenLayers-API, kannst jede Karte lesen und übst zum Spaß
  EPSG-Codes? Beste Voraussetzungen den Vorjahressieger zu entthronen!%
}%


%%%%%%%%%%%%%%%%%%%%%%%%%%%%%%%%%%%%%%%%%%%
\sponsorBoxA{102-nti_cwsm.png}{0.38\textwidth}{3}{%
\textbf{Silbersponsor, Aussteller}\\
Die NTI CWSM GmbH mit Standorten in Magdeburg, Berlin, Dresden und Mannheim ist
im Bereich CAD, GIS, BIM und Computervernetzungen tätig.

Wir sind Entwickler der SAGis-Produktfamilie und bieten ein komplettes
Portfolio für die Erfassung, Aufbereitung und Präsentation von raumbezogenen
Daten.

Mit SAGis web und Geoportalen wird die Bereitstellung raumbezogener
Daten im Browser ermöglicht. Kommunale Daten werden von Anwendern mit SAGis
Kommunal verwaltet. Im Bereich der Netzdokumentation unterstützen wir Sie bei
der Dokumentation von Trink- und Abwassernetzen. Weiterhin stellen wir
leistungsfähige GIS-Lösungen für die mobile Nutzung zur Verfügung.
}


\noindent\sponsorBoxA{103-geoinfo}{0.38\textwidth}{2}{%
\textbf{Silbersponsor}\\
\noindent Die GEOINFO IT AG verbindet die Leidenschaft für
Geoinformationen mit der Begeisterung für Softwaretechnologien. Aus den
Ideen unserer Kunden entstehen innovative Lösungen. Darunter
bedürfnisgerechte Fachanwendungen für Infrastruktur, Sicherheit,
Vegetation und Landwirtschaft. Wir machen umfangreiche Geoinformationen
einfach und zielgruppengerecht zugänglich.

Das Rückgrat unserer Geodateninfrastrukturen bilden eigene moderne
Rechenzentren. Als einer der einzigen Schweizer Anbieter der Branche
entwickeln wir zudem selbständig Software-Produkte. Auch deswegen gehört
die GEOINFO Applications AG in der Schweiz zu den führenden Anbietern
umfassender Geodateninfrastrukturen.%
}


\sponsorBoxA{104-regiodata}{0.47\textwidth}{5}{%
\textbf{Silbersponsor, Aussteller}\\
regioDATA bietet als Dienstleister komplette Lösungen rund um
Geoinformationssysteme (GIS) für Ver- und Entsorgungsnetze, sowie die
Planung und Bauleitung für Energieversorger und Kommunen an.  Raumbezogene
Informationssysteme sind seit 1997 das Spezialgebiet des Unternehmens.
Umfassender Service, Ingenieurvermessung, lückenlose Dokumentation und
Planungsleistungen, sowie die Software-Entwicklung sind die Eckpfeiler der
Kundenbetreuung.  Eine ausgereifte Produktfamilie mit Webbasierten
Kartendiensten und Applikationen für Fachanwender runden das Angebot ab.
regioDATA ist Teil der badenovaGRUPPE mit derzeit ca. 130 Mitarbeitenden.
}


\sponsorBoxA{105-con4gis-small}{0.53\textwidth}{5}{%
\textbf{Silbersponsor}\\
con4gis ist der GIS"=Baukasten für das beliebte Content"=Management"=System Contao. Seit 2010 wird er in Jever von der Küstenschmiede entwickelt. Contao und con4gis sind Open Source auf GitHub.

Die aktuelle Contao LTS Version 4.9 und die neue con4gis 7 aus Februar 2020 sind gute Einstiegsmöglichkeiten in beide Systeme. con4gis umfasst eine große Bibliothek an Symfony-Bundles zum Aufbau von Webanwendungen. Die Kartentechnik nutzt OpenLayers 6 und andere etablierte Open-Source- und Open-Data-Technologien. Contao und con4gis sind aktuell dokumentiert.

Mehr zur Technik und zur aktuellen Version findet Ihr in Eurer Konferenztasche oder auf con4gis.org.
}


\sponsorBoxA{106-db-mindbox.pdf}{0.57\textwidth}{3}{%
\textbf{Silbersponsor}\\
GoBeta.de ist die offene Innovations"=Plattform, auf der wir uns mit
Startups, Partnern und Unterstützern austauschen, zusammenarbeiten und
neue Ideen vorstellen.

Für Entwicklerinnen, Techies, Hackerinnen und Mobilitätsbegeisterte
Tester\allowbreak*innen bieten wir eine Community"=Plattform mit Daten,
Codes, Anleitungen und Infos zu echten Projekten.

Bringt Eure Ideen ein, sucht Unterstützung und lernt Euch online und im
echten Leben kennen, z.\,B. auf Hackathons und MeetUps. Neben Zugang zu
Daten von DB und Partner könnt Ihr hier auch Feedback von
Nutzer\allowbreak*innen bekommen. Gemeinsam gestalten wir die Zukunft der
Mobilität als Teil der Digitalisierungsaktivitäten der DB. \#powered by DB
mindbox und Euch!
}


\sponsorBoxA{201-mundialis}{0.42\textwidth}{3}{%
\textbf{Bronzesponsor}\\
mundialis ist spezialisiert auf die Auswertung und Verarbeitung von
Fernerkundungs- und Geodaten mit dem Schwerpunkt Cloud-basierte
Geoprozessierung. Wir setzen Freie und
Open"=Source"=Geoinformationssysteme (GRASS GIS, actinia, QGIS, u.\,a.)
ein, mit denen wir maßgeschneiderte Lösungen für den Kunden entwickeln.
}


\sponsorBoxA{202-wagner-it-small}{0.35\textwidth}{3}{%
\textbf{Bronzesponsor}\\
Wagner-IT ist seit 2011 für kleinere Gemeinden und Städte im Bereich der
GIS- und Web-GIS"=Betreuung sowie des Geodatenmanagements tätig.
Realisiert werden individuelle Lösung auf Basis von QGIS, QGIS-Server,
PostgreSQL und Web-Clients (z.\,B. dem Lizmap-Client). https://wagner-it.de/
}


\sponsorBoxA{203-geoinformatik-buero-dassau}{0.26\textwidth}{2}{%
\textbf{Bronzesponsor}\\
Die Geoinformatikbüro Dassau GmbH aus Düsseldorf bietet seit 2006
Beratung, Konzeption, Schulung, Wartung, Support und Programmierung zum
Thema GIS und GDI auf Open-Source-Basis. Ein Fokus liegt auf der Software
QGIS, QGIS Server, QGIS Web Client, GBD WebSuite, Postgres/PostGIS und
GRASS GIS.
}


\sponsorBoxA{204-gkg-kassel}{0.40\textwidth}{6}{%
\textbf{Bronzesponsor, Aussteller}\\
GKG Kassel, Dr.-Ing. Claas Leiner~--
Schulungen, Dienstleistungen und Sup\-port rund um QGIS, GRASS, Spa\-tia\-Lite und PostGIS. Mit
angepassten QGIS"=Oberflächen, dem QGIS"=Modeller sowie SpatialSQL strukturiere ich Ihre Geodaten und
ermögliche maßgeschneiderte Analyse und Präsentation mit freier Software.%
}


\sponsorBoxA{205-disy}{0.40\textwidth}{4}{%
\textbf{Bronzesponsor}\\
Wir verbinden Datenanalytik und Geoinformation auf innovative Weise. Basis
hierfür ist unsere Kerntechnologie Cadenza. Gemeinsam mit mehr als 110
Mitarbeitern an unserem Standort in Karlsruhe arbeiten wir tagtäglich mit
Begeisterung und Kreativität an individuellen Lösungen für unsere Kunden.
}


\sponsorBoxA{206-bemastergis-neu}{0.53\textwidth}{5}{%
  \textbf{Bronzesponsor,\linebreak
    Aussteller}\\
  GIS-Anwender brauchen Fachwissen, stellen sich der schnellen Soft- und
  Hardware-Entwicklung, wollen Führungsaufgaben übernehmen.  Die
  Hochschule Anhalt bietet hierzu ein für fünf Semester konzipiertiertes,
  auf Anwender aus kommunaler Verwaltung, Planung, Umweltschutz
  zugeschnittenes Online"=Masterstudium GIS.
}


\noindent\sponsorBoxA{207-mapwebbing}{0.42\textwidth}{3}{%
\textbf{Bronzesponsor}\\
mapwebbing, gegr. 2008 von Lars Lingner, Dipl.-Inf. (FH), spezialisiert auf
die Planung und Erstellung individueller Kartenanwendungen. Kernkompetenz
ist die Entwicklung maßgeschneiderter Konzepte zur Verarbeitung von
Geodaten sowie die Aufbereitung von Kartenmaterial für den professionellen
Druck.%
}


\sponsorBoxA{208-geofabrik}{0.53\textwidth}{3}{%
  \textbf{Bronzesponsor}\\
  Die Geofabrik bietet die Datenaufbereitung von OpenStreetMap"=Daten an,
  betreibt OpenStreetMap-basierte Serverdienste und hilft Ihnen bei der
  Installation eigener Karten-, Geocoding- oder Routingserver.
}


\sponsorBoxA{209-intevation}{0.42\textwidth}{4}{%
  \textbf{Bronzesponsor}\\
  Die Intevation GmbH ist seit 20 Jahren ein unabhängiger IT"=Dienstleister
  aus Osnabrück mit 25 Mitarbeitenden. Ob Planung, Umsetzung oder
  Weiterentwicklung von bestehenden und neuen Software-Lösungen~-- wir
  setzen konsequent auf Freie Software (Open Source). Damit Sie
  unabhängig bleiben!
}


\sponsorBoxA{301-vag}{0.52\textwidth}{3}{%
  \textbf{Mobilitäts-Sponsor}\\
  Die Freiburger Verkehrs AG ist die Anbieterin der vernetzten und
  klimafreundlichen Mobilität in Freiburg. Rund 80,5 Mio. Fahrgäste im Jahr
  nutzen ihre Stadtbahnen, Busse und Leihräder. So leistet die VAG einen
  wichtigen Beitrag zur Verbesserung der Mobilität in Freiburg~-- auch mit der
  App VAG mobil, mit der die Kombination von verschiedenen Mobilitätsangeboten
  noch einfacher und schneller werden soll. www.vag-freiburg.de
}


\sponsorBoxA{401-tib.png}{0.630\textwidth}{4}{%
\textbf{Medienpartner}\\
Die Technische Informationsbibliothek in Hannover ist die Deutsche
Zentrale Fachbibliothek für Technik und Naturwissenschaften. Sie versorgt in
ihren Spezialgebieten die nationale wie internationale Forschung und Industrie
mit Literatur und Information in gedruckter und elektronischer Form. Mit dem
AV-Portal bietet die TIB unter av.tib.eu eine Online-Plattform für
wissenschaftliche Videos.%
}


\sponsorBoxA{images-print/402-voc.pdf}{0.27\textwidth}{3}{%
  \textbf{Medienpartner C3VOC}\\
  \noindent Die Videoaufzeichnung dieser Konferenz erfolgt durch die Freiwilligen des
  CCC \mbox{Video} Operation Center.
}


\sponsorBoxA{403-osgeolive.pdf}{0.55\textwidth}{3}{%
\textbf{Medienpartner}\\
OSGeoLive bietet Ihnen die Möglichkeit, freie und Open"=Source"=Software in Verbindung mit Geodaten
auszuprobieren, ohne aufwendige Installationen durchführen zu müssen. Das OSGeoLive"=Projekt wird von
der OSGeo Foundation und vielen weiteren Aktiven getragen.%
}


\RaggedRight
\sponsorBoxA{404-yaga.png}{0.44\textwidth}{3}{%
\textbf{Medienpartner}\\
\noindent Das YAGA"=Development"=Team entwickelt die FOSSGIS-App für Android und iOS.
}
\justifying

