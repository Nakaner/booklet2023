
% time: Friday 09:00
% URL: https://pretalx.com/fossgis2023/talk/fossgis2023-23868-das-beste-der-60er-70er-und-80er-hochauflsende-spionagesatellitenaufnahmen/

%
\newTimeslot{09:00}
\noindent\abstractHSeins{%
  Markus Metz%
}{%
  Das Beste der 60er, 70er und 80er: hochauflösende Spionagesatellitenaufnahmen%
}{%
}{%
  Aufnahmen der historischen Spionagesatelliten aus den "`Keyhole"' Missionen der 60er, 70er, und 80er
  Jahre sind aufgrund ihrer Qualität einzigartig, allerdings erfordern sie eine besondere Methodik
  zur Entzerrung der Panorama-Aufnahmen. Hier stellen wir die Ergebnisse der Orthorektifizierung
  dieser Aufnahmen mit der Open Source Software GRASS GIS vor.%
}%


%%%%%%%%%%%%%%%%%%%%%%%%%%%%%%%%%%%%%%%%%%%

% time: Friday 09:00
% URL: https://pretalx.com/fossgis2023/talk/fossgis2023-23866-geonode-in-forschungsdateninfrastrukturen/

%

\noindent\abstractHSzwei{%
  Benedikt Gräler, Henning Bredel%
}{%
  GeoNode in Forschungsdateninfrastrukturen%
}{%
}{%
  In verschiedenen fachlichen und technischen Kontexten haben wir GeoNode als zentrale Komponente
  von Forschungsdateninfrastrukturen eingesetzt. In den einzelnen Szenarien wurden verschiedene
  Weiterentwicklungen und Ergänzungen vorgenommen. Dazu gehören die Anbindung von Remote-Diensten
  über offene OGC Schnittstellen oder die Integration von Cloud-optimized-GeoTiffs (COG). Zur
  Darstellung der neuen Datentypen und spezialisierten Fachinformationen wurden dedizierte
  Dashboards entwickelt.%
}%


%%%%%%%%%%%%%%%%%%%%%%%%%%%%%%%%%%%%%%%%%%%

% time: Friday 09:00
% URL: https://pretalx.com/fossgis2023/talk/fossgis2023-23791-kartieren-im-hochgebirge-ein-praxisbericht/

%

\noindent\abstractHSdrei{%
  Mathias Gröbe%
}{%
  Kartieren im Hochgebirge~-- ein Praxisbericht%
}{%
}{%
  Für eine neue Alpenvereinskarte rund um den Berg Ushba im Großen Kaukasus waren Feldarbeiten im
  Hochgebirge notwendig, wobei die gesammelten Daten in OpenStreetMap eingetragen wurden. Dafür
  wurde eine Methodik entwickelt, um möglichst zielgerichtet die fehlenden Wege und Namen zu
  sammeln, zu validieren und zu dokumentieren.%
}%


%%%%%%%%%%%%%%%%%%%%%%%%%%%%%%%%%%%%%%%%%%%

% time: Friday 09:30
% URL: https://pretalx.com/fossgis2023/talk/fossgis2023-23715-automatisierte-detektion-von-baumstandorten-in-der-metropole-ruhr/

%
\newTimeslot{09:30}
\noindent\abstractHSeins{%
  Anika Weinmann, Lina Krisztian, Markus Metz%
}{%
  Automatisierte Detektion von Baumstandorten in der Metropole Ruhr%
}{%
}{%
  In diesem Vortrag wird die automatisierte Detektion von Baumstandorten mit Hilfe von machinellem
  Lernen und Fernerkundungsdaten, die im Zuge eines Projektes mit dem Regionalverband Ruhr (RVR)
  entwickelt wurde, präsentiert.%
}%


%%%%%%%%%%%%%%%%%%%%%%%%%%%%%%%%%%%%%%%%%%%

% time: Friday 09:30
% URL: https://pretalx.com/fossgis2023/talk/fossgis2023-23910-gis-datenstrme-stream-processing-mit-apache-streampipes/

%

\noindent\abstractHSzwei{%
  Florian Micklich%
}{%
  GIS \& Datenströme;  Stream Processing mit Apache StreamPipes%
}{%
}{%
  Anhand Open-Source-Software Apache StreamPipes wird die praktische Anwendung von Geo-Operatoren
  beim Stream-Processing vorgestellt. Zudem werden die allgemeinen Grundlagen und die
  Herausforderungen von Stream-Processing erklärt und mit den klassischen Prinzipien des
  Batch-Processing verglichen. Auf Themen wie Interoperabilität durch Umsetzung von Standards als
  auch Ontologie im Geo-Bereich wird ebenfalls eingegangen.%
}%


%%%%%%%%%%%%%%%%%%%%%%%%%%%%%%%%%%%%%%%%%%%

% time: Friday 09:30
% URL: https://pretalx.com/fossgis2023/talk/fossgis2023-23822-verwendung-von-osm-daten-zur-kartierung-des-urbanen-ffentlichen-raums/

%

\noindent\abstractHSdrei{%
  Ester Scheck%
}{%
  Verwendung von OSM-Daten zur Kartierung des urbanen, öffentlichen Raums%
}{%
}{%
  Inspiriert von der [Nolli-Karte](https://www.lib.berkeley.edu/EART/maps/nolli.html) sollen
  öffentliche Räume auf Basis aktueller OpenStreetMap Daten identifiziert und kartiert werden. Dazu
  entwickle ich ein Skript zur automatisierten Geodatenverarbeitung, welches auf einen Stadtteil von
  Wien angewendet werden soll.%
}%


%%%%%%%%%%%%%%%%%%%%%%%%%%%%%%%%%%%%%%%%%%%

% time: Friday 10:00
% URL: https://pretalx.com/fossgis2023/talk/fossgis2023-23889-prozessierung-von-uas-befliegungen-automatisiert-gedacht/

%
\newTimeslot{10:00}
\noindent\abstractHSeins{%
  Christian Bauer, Martin Weis%
}{%
  Prozessierung von UAS-Befliegungen automatisiert gedacht%
}{%
}{%
  UAS werden in der Landwirtschaft u. a. eingesetzt, um ein exaktes Bild vom Zustand der Pflanzen
  und Böden zu erhalten. Unterschiedliche Sensoren und hohe Auflösungen steigern die Anforderungen
  an Verfahren zur Auswertung. Wir zeigen, wie anhand kommandozeilenbasierter
  Automatisierungswerkzeuge Projekte vereinfacht abgearbeitet werden können und welche zentrale
  Rolle dabei Open Source Software spielt, um große Datenmengen und wiederkehrende Projekte
  automatisert zu verarbeiten.%
}%


%%%%%%%%%%%%%%%%%%%%%%%%%%%%%%%%%%%%%%%%%%%

% time: Friday 10:00
% URL: https://pretalx.com/fossgis2023/talk/fossgis2023-23848-signalo-erhebung-und-darstellung-von-strassenschildern-mit-qgis/

%

\noindent\abstractHSzwei{%
  Isabel Kiefer%
}{%
  SIGNALO~-- Erhebung und Darstellung von Strassenschildern mit QGIS%
}{%
}{%
  SIGNALO ist ein Gemeinschaftsprojekt von mehreren Schweizer Städten und einem Kanton. Es besteht
  aus einem PostgreSQL-Datenmodell und einer Arbeitsumgebung für die Software QGIS und QField im
  Bereich der vertikalen Beschilderung und Signalisation. Der Aufbau des Moduls, das Arbeitsvorgehen
  mit öffentlichen Partnern und die Herausforderungen bei der Darstellung komplexer Signale werden
  vorgestellt.%
}%


%%%%%%%%%%%%%%%%%%%%%%%%%%%%%%%%%%%%%%%%%%%

% time: Friday 10:00
% URL: https://pretalx.com/fossgis2023/talk/fossgis2023-23440-trinkwasser-und-trinkwasser-orte-mapping/

%

\noindent\abstractHSdrei{%
  Annika%
}{%
  Trinkwasser und Trinkwasser-Orte Mapping%
}{%
}{%
  Öffentliche Trinkstellen sind wichtige Komponenten des Hitze- und Umweltschutzes~-- damit sie auch
  gefunden werden- haben wir von a tip: tap im Juni 2022 das Trink-Orte OSM Projekt gestartet. Ich
  möchte über den Einstiegsprozess, Anknüpfungspunkte, Erkenntnisse, Ergebnisse und
  Herausforderungen sprechen und natürlich über die Vorteile von Leitungswasser ;)%
}%


%%%%%%%%%%%%%%%%%%%%%%%%%%%%%%%%%%%%%%%%%%%

% time: Friday 11:00
% URL: https://pretalx.com/fossgis2023/talk/fossgis2023-23885-3d-tiles-anwendertreffen/

%
\newTimeslot{11:00}
\noindent\abstractExpBoFAnw{%
  Pirmin Kalberer%
}{%
  3D Tiles Anwendertreffen%
}{%
}{%
  Die 3D-Tiles Spezifikation wird bis zur FOSSGIS-Konferenz voraussichtlich als neue OGC Community
  Standard Version veröffentlich sein. Dieses Treffen soll Anwender und Interessierte
  zusammenbringen, um Erfahrungen und Tipps auszutauschen.%
}%


%%%%%%%%%%%%%%%%%%%%%%%%%%%%%%%%%%%%%%%%%%%

% time: Friday 11:00
% URL: https://pretalx.com/fossgis2023/talk/fossgis2023-23776-ad-hoc-qgis-plugin-entwicklung-zur-bewertung-der-radiologischen-lage-im-ukrainekrieg/

%

\noindent\abstractHSeins{%
  Dr. Marco Lechner%
}{%
  Ad hoc QGIS-Plugin Entwicklung zur Bewertung der radiologischen Lage im Ukrainekrieg%
}{%
}{%
  Das Bundesamt für Strahlenschutz bewertet die radiologische Situation auch beim Krieg in der
  Ukraine, bei dem es immer wieder zu Kampfhandlungen um radioaktive Anlagen kommt. Dabei wurde das
  BfS vor eine besondere Herausforderung gestellt, da wichtige Messdaten nur über nicht vorher
  vereinbarte und nicht standardisierte Formate und Datenwege, zugreifbar sind. Durch die ad hoc
  Entwicklung von QGIS-Plugins und ein komplexes QGIS-Kartenprojekt wurde die tägliche Bewertung der
  Messdaten möglich.%
}%


%%%%%%%%%%%%%%%%%%%%%%%%%%%%%%%%%%%%%%%%%%%

% time: Friday 11:00
% URL: https://pretalx.com/fossgis2023/talk/fossgis2023-23917-neues-von-der-openrailwaymap/

%

\noindent\abstractHSzwei{%
  Michael Reichert%
}{%
  Neues von der OpenRailwayMap%
}{%
}{%
  Die OpenRailwayMap, eine ausschließlich auf OpenStreetMap-Daten basierende
  Eisenbahn-Infrastrukturkarte wird bald zehn Jahre alt. Der Vortrag zeigt, was sich in den letzten
  Monaten beim Projekt getan hat. Ein Blick auf die "`Fahrgastzahlen"' wird auch nicht fehlen.%
}%


%%%%%%%%%%%%%%%%%%%%%%%%%%%%%%%%%%%%%%%%%%%

% time: Friday 11:00
% URL: https://pretalx.com/fossgis2023/talk/fossgis2023-23733-how-far-how-much-how-many-hilbert-und-dijkstra-zum-appell-/

%

\noindent\abstractHSdrei{%
  Matthias Daues%
}{%
  How far, how much, how many~--  Hilbert und Dijkstra zum Appell!%
}{%
}{%
  Vertriebspotential zu ermitteln ist eine Kernfunktion von GIS im Unternehmenskontext.
  Häufig spielen dabei Graph-Datenmodelle und Routing-Algorithmen eine zentrale Rolle.
  Mit PostgreSQL, PostGIS, pgRouting, Python und FME gelingen blitzschnelle Antwortmaschinen auf
  solche und ähnliche Fragen.%
}%


%%%%%%%%%%%%%%%%%%%%%%%%%%%%%%%%%%%%%%%%%%%

% time: Friday 11:00
% URL: https://pretalx.com/fossgis2023/talk/fossgis2023-23584-sozialhelden-wie-barrierefrei-ist-unser-planet-/

%

\noindent\abstractHSvier{%
  Björn Uhlig, Sebastian Felix Zappe%
}{%
  Sozialhelden: Wie barrierefrei ist unser Planet?%
}{%
}{%
  Was für Daten findet man über Barrierefreiheit auf OpenStreetMap~-- und wie hilft das Menschen mit
  Behinderung überhaupt?%
}%


%%%%%%%%%%%%%%%%%%%%%%%%%%%%%%%%%%%%%%%%%%%

% time: Friday 11:05
% URL: https://pretalx.com/fossgis2023/talk/fossgis2023-23691-get-your-own-openstreetmap-dataset-running-in-a-geoserver-instance-in-2-steps/

%
\newTimeslot{11:05}
\noindent\abstractHSzwei{%
  José Eduardo Macchi%
}{%
  Get your own OpenStreetMap dataset running in a Geoserver instance in 2 steps%
}{%
}{%
  Get your preferred OSM dataset (ie. country) running in a local Geoserver instance with only 2
  commands and avoid any dependence on an external provider.
  Simple, fast, clean solution. Lowering the barrier to entry to geospatial technology use and
  development.%
}%


%%%%%%%%%%%%%%%%%%%%%%%%%%%%%%%%%%%%%%%%%%%

% time: Friday 11:10
% URL: https://pretalx.com/fossgis2023/talk/fossgis2023-23908-psychische-hilfsangebote-mappen-kooperation-mit-dem-mut-atlas/

%
\newTimeslot{11:10}
\noindent\abstractHSzwei{%
  Sebastian Burger%
}{%
  Psychische Hilfsangebote mappen: Kooperation mit dem MUT-ATLAS%
}{%
}{%
  Der MUT-ATLAS ist Deutschlands erste Online-Karte der Versorgungs- und Selbsthilfe-Landschaft im
  Bereich psychische Gesundheit, die den Anspruch hat, über alle Bundeslandgrenzen und Berufsgruppen
  hinweg Hilfsangebote aufzuzeigen. Um neben bzw. vor (..) der Einspeisung von Datensätzen durch
  zentrale Ärzte- und Therapeutendatenbanken möglichst schnell erste Facilities darzustellen, könnte
  eine Kooperation mit OSM helfen, schneller relevant zu werden.
  https://www.mut-atlas.de%
}%


%%%%%%%%%%%%%%%%%%%%%%%%%%%%%%%%%%%%%%%%%%%

% time: Friday 11:15
% URL: https://pretalx.com/fossgis2023/talk/fossgis2023-23873-digitale-kartendaten-fr-alle/

%
\newTimeslot{11:15}
\noindent\abstractHSzwei{%
  Julian Striegl%
}{%
  Digitale Kartendaten für alle%
}{%
}{%
  Welche Daten und Lösungen braucht es, um Gebäudeinformationen und digitale Karten barrierefrei
  zugänglich zu machen?%
}%


%%%%%%%%%%%%%%%%%%%%%%%%%%%%%%%%%%%%%%%%%%%

% time: Friday 11:30
% URL: https://pretalx.com/fossgis2023/talk/fossgis2023-23879-open-geodata-gi-software-und-science-am-beispiel-einer-rumlichen-covid-studie/

%
\newTimeslot{11:30}
\noindent\abstractHSeins{%
  Tobia Lakes%
}{%
  Open geodata, GI-software und science am Beispiel einer räumlichen COVID-Studie%
}{%
}{%
  Open science Initiativen and FAIR data Prinzipien haben in den letzten Jahren an großer Bedeutung
  in Forschungsarbeiten gewonnen. Ziel des Beitrags ist es am Beispiel eines laufenden
  Forschungsprojektes zur kleinräumigen Analyse von COVID-19-Fällen in Berlin darzustellen, wie Open
  Science Ansätze in Forschungsprojekten datenschutzkonform berücksichtigt werden können.%
}%


%%%%%%%%%%%%%%%%%%%%%%%%%%%%%%%%%%%%%%%%%%%

% time: Friday 11:30
% URL: https://pretalx.com/fossgis2023/talk/fossgis2023-23748-irische--steine-in-osm-und-wikidata/

%

\noindent\abstractHSzwei{%
  Florian Thiery%
}{%
  Irische Ogham Steine in OSM und Wikidata%
}{%
}{%
  Ogham-Steine sind mit der frühmittelalterlichen Ogham-Schrift versehene Monolithen, die
  vor allem in Irland, zwischen dem 4. und 9. Jahrhundert errichtet wurden. Viele von ihnen sind in
  "`der freien Wildbahn” oder in Museen für die "`Volunteer Community” einsehbar. Dieser Vortrag
  beschreibt die Dokumentation, Modellierung und Veröffentlichung dieser archäologischen Fundgattung
  in Community Hubs wie OSM und Wikidata mit der Nutzung von Linked Open Data Technologien.%
}%


%%%%%%%%%%%%%%%%%%%%%%%%%%%%%%%%%%%%%%%%%%%

% time: Friday 11:30
% URL: https://pretalx.com/fossgis2023/talk/fossgis2023-23550-persistente-identifikatoren-fr-open-source-gis-best-practices-und-bleeding-edge/

%

\noindent\abstractHSdrei{%
  Peter Löwe, Ralf Löwner%
}{%
  Persistente Identifikatoren für Open Source GIS: Best Practices und Bleeding Edge%
}{%
}{%
  Dieser Werkstattbericht gibt einen aktuellen Überblick zum bereits erreichten Stand bei der
  Einführung von persistenten Identifikatoren (PID) und dem resultierenden praktischen Nutzen für
  Anwender:innen und Entwickler:innen durch die OSGeo-Projekte, sowie eine Vorschau auf die nächsten
  Herausforderungen und Nutzen durch neue Einsatzszenarios für die Projektcommunities und speziell
  Code Committer die in der wissenschaftlichen Forschung und Lehre aktiv sind, sowie Studierende.%
}%


%%%%%%%%%%%%%%%%%%%%%%%%%%%%%%%%%%%%%%%%%%%

% time: Friday 12:00
% URL: https://pretalx.com/fossgis2023/talk/fossgis2023-23901-aufbau-einer-plattform-fr-die-risikobewertung-von-biodiversittsportfolios/

%
\newTimeslot{12:00}
\noindent\abstractHSeins{%
  Markus Eichhorn%
}{%
  Aufbau einer Plattform für die Risikobewertung von Biodiversitätsportfolios%
}{%
}{%
  Die neue EU-Taxonomie für nachhaltige Finanzen erfordert neue Methoden und Instrumente, um eine
  Bewertung von Unternehmen oder ganzen Portfolios im Hinblick auf Ökosystemleistungen und
  Naturkapitalwerte zu ermöglichen. Wir entwickeln eine prototypische Plattform, die es ermöglicht,
  Risiken in Bezug auf die biologische Vielfalt in Portfolios zu bewerten. Dabei werden Earth
  Observation Datenprodukte sowie Lieferketten- und Anlagendaten von börsennotierten Unternehmen
  genutzt.%
}%


%%%%%%%%%%%%%%%%%%%%%%%%%%%%%%%%%%%%%%%%%%%

% time: Friday 12:00
% URL: https://pretalx.com/fossgis2023/talk/fossgis2023-23788-strategie-und-wunschzettel/

%

\noindent\abstractHSzwei{%
  Dr. Roland Olbricht%
}{%
  Strategie und Wunschzettel%
}{%
}{%
  Zum rasanten Wachstum des OpenStreetMap-Projekts wird es ja Pläne und genug Geld geben, das
  weiterzuentwickeln. Wirklich?
  Mehrere Versuche, eine Strategie zu definieren, sind eingeschlafen, während unverbindliche
  Wunschlisten gerne aufgestellt werden und es für das eine oder andere Feature Ungeduld gibt. An
  alles wesentliche Existierende und Gewünschte soll ein ungefähres Preischild gehängt werden, um
  abwägen zu können, was Priorität verdienen könnte oder zumindest realisierbar ist.%
}%


%%%%%%%%%%%%%%%%%%%%%%%%%%%%%%%%%%%%%%%%%%%

% time: Friday 12:00
% URL: https://pretalx.com/fossgis2023/talk/fossgis2023-23877-open-geodata-and-software-im-hochschulstudium-der-geographie/

%

\noindent\abstractHSdrei{%
  Tobia Lakes%
}{%
  Open Geodata and -software im Hochschulstudium der Geographie%
}{%
}{%
  Ziel dieses Beitrags ist eine Vorstellung der Einbindung von Open Geodata und -Software in das
  Bachelor- und Masterstudium am Beispiel des Geographiestudiums für Mono- und Lehramtsstudierende
  an der Humboldt-Universität. Der Beitrag lädt zur Diskussion über Ansätze und Vorteile der
  Einbindung von Open Geodata und -Software im Hochschulstudium ein.%
}%


%%%%%%%%%%%%%%%%%%%%%%%%%%%%%%%%%%%%%%%%%%%

% time: Friday 12:00
% URL: https://pretalx.com/fossgis2023/talk/fossgis2023-23557-agil-erreichbar-erreichbarkeitsanalysen-fr-deutschland/

%

\noindent\abstractHSvier{%
  Andreas Weiner%
}{%
  Agil erreichbar~-- Erreichbarkeitsanalysen für Deutschland%
}{%
}{%
  Drei Bundesbehörden und agile Softwareentwicklung mit open source Komponenten; geht das?
  Ein kurzer Erfahrungsbericht, dass es geht; man muss es aber wirklich wollen.%
}%


%%%%%%%%%%%%%%%%%%%%%%%%%%%%%%%%%%%%%%%%%%%

% time: Friday 12:05
% URL: https://pretalx.com/fossgis2023/talk/fossgis2023-23861-smartphone-imu-und-landmark-basierte-indoor-positionierung/

%
\newTimeslot{12:05}
\noindent\abstractHSvier{%
  Christian Willmes%
}{%
  Smartphone IMU- und Landmark-basierte Indoor Positionierung%
}{%
}{%
  Der Beitrag beschreibt die Architektur und Aufbau eines auf Smartphone-Sensoren basierenden Indoor
  Positionierungsansatzes.%
}%


%%%%%%%%%%%%%%%%%%%%%%%%%%%%%%%%%%%%%%%%%%%

% time: Friday 12:10
% URL: https://pretalx.com/fossgis2023/talk/fossgis2023-23798-ogc-client-ab-jetzt-mit-support-fr-ogc-apis-/

%
\newTimeslot{12:10}
\noindent\abstractHSvier{%
  Olivia Guyot%
}{%
  OGC-client, ab jetzt mit Support für OGC APIs!%
}{%
}{%
  OGC-client ist eine einfach zu verwendende und leichte Bibliothek, die für die Interaktion mit
  OGC-Diensten konzipiert wurde. Unterstützungen für neue Protokolle werden laufend implementiert,
  um mit Neuentwicklungen der OGC-API Schritt zu halten!%
}%


%%%%%%%%%%%%%%%%%%%%%%%%%%%%%%%%%%%%%%%%%%%

% time: Friday 12:15
% URL: https://pretalx.com/fossgis2023/talk/fossgis2023-23721-oss-im-schweizerischen-reb-kataster-erfahrungen-und-technische-herausforderungen/

%
\newTimeslot{12:15}
\noindent\abstractHSvier{%
  Wolfgang Kaltz, Marion Baumgartner%
}{%
  OSS im schweizerischen ÖREB-Kataster: Erfahrungen und technische Herausforderungen%
}{%
}{%
  Die OpenOereb Community hat das Ziel, gemeinschaftliche OSS-Lösungen für den Betrieb des
  schweizerischen ÖREB-Katasters zu entwickeln und zu pflegen. In diesem Vortrag wird erläutert, wie
  sich die Community seit 2017 weiterentwickelt hat und welche Herausforderungen im produktiven
  Betrieb aufgetreten sind, z.B. bezüglich geometrischer Ungenauigkeiten mit Shapely / PostGIS.%
}%


%%%%%%%%%%%%%%%%%%%%%%%%%%%%%%%%%%%%%%%%%%%

% time: Friday 14:00
% URL: https://pretalx.com/fossgis2023/talk/fossgis2023-23706-16-jahre-fossgis-und-osm/

%
\newTimeslot{14:00}
\noindent\abstractHSeins{%
  Jochen Topf%
}{%
  16 Jahre FOSSGIS und OSM%
}{%
}{%
  Vor 16 Jahren gab es schon einmal eine FOSSGIS in Berlin-Adlershof. Und dort habe ich meinen
  ersten Vortrag über OpenStreetMap gehalten, der erste OSM-Vortrag auf einer FOSSGIS überhaupt. Und
  der Beginn einer wunderbaren Freundschaft zwischen FOSSGIS und OSM. Anlass genug in den DeLorean
  zu steigen, zu schauen was damals war, was heute ist und was dazwischen passierte.%
}%


%%%%%%%%%%%%%%%%%%%%%%%%%%%%%%%%%%%%%%%%%%%

% time: Friday 14:30
% URL: https://pretalx.com/fossgis2023/talk/fossgis2023-23568-fossgis-jeopardy/

%
\newTimeslot{14:30}
\noindent\abstractHSeins{%
  Johannes Kröger, Tobia Lakes%
}{%
  FOSSGIS-Jeopardy%
}{%
}{%
  Das FOSSGIS-Jeopardy bietet wieder spannende Fragen zu (mehr oder weniger) wissenswerten Fakten
  und vor allem viel Spaß für jung und alt, alt und neu.%
}%


%%%%%%%%%%%%%%%%%%%%%%%%%%%%%%%%%%%%%%%%%%%

% time: Friday 15:30
% URL: https://pretalx.com/fossgis2023/talk/fossgis2023-25282-abschlussveranstaltung/

%
\newTimeslot{15:30}
\noindent\abstractHSeins{%
  FOSSGIS e.V.%
}{%
  Abschlussveranstaltung%
}{%
}{%
  Drei spannende Konferenztage gehen zu Ende. Ein gemeinsamer Abschluss soll erfolgen mit Rückblick
  auf die Konferenz und das erlebte. Natürlich auch mit einem Ausblick auf das nächste Jahr 2024.%
}%


%%%%%%%%%%%%%%%%%%%%%%%%%%%%%%%%%%%%%%%%%%%

% time: Friday 16:00
% URL: https://pretalx.com/fossgis2023/talk/fossgis2023-25281-sektempfang/

%
\newTimeslot{16:00}
\noindent\abstractHSeins{%
  FOSSGIS e.V.%
}{%
  Sektempfang%
}{%
}{%
  Der FOSSGIS e.V. lädt alle Mitglieder des FOSSGIS-Vereins, Freunde und Interessierte ab 16:00 Uhr
  herzlich zum Sektempfang zum Ausklang der FOSSGIS 2023 am FOSSGIS-Vereins-Stand ein.%
}%


%%%%%%%%%%%%%%%%%%%%%%%%%%%%%%%%%%%%%%%%%%%
