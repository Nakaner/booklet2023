
% time: Thursday 07:00
% URL: https://pretalx.com/fossgis2020/talk/WFPTSP/
\renewcommand{\conferenceDay}{\donnerstag}
%
\newTimeslot{07:00}
\noindent\abstractOther{%
  Marco Lechner%
}{%
  Sportliche Stadtführung%
}{%
}{%
  Um richtig in Schwung zu kommen, werden wir in 3 verschiedenen Runden zu je 30 Minuten laufend die
  sonnigste Stadt Deutschlands und den Schwarzwaldrand erkunden. Je nach Laufniveau gibt es die
  Möglichkeit früher oder später ein- bzw. auszusteigen.%
}%
{%
  Rahmenprogramm%
}%



%%%%%%%%%%%%%%%%%%%%%%%%%%%%%%%%%%%%%%%%%%%

% time: Thursday 09:00
% URL: https://pretalx.com/fossgis2020/talk/ZXTALA/

%
\newSmallTimeslot{09:00}
\noindent\abstractHSAnatomie{%
  Andreas Rabe%
}{%
  Visualisierung und Analyse von Satelliten\-bildern mit der EnMAP-Box%
}{%
}{%
  Die EnMAP-Box ist ein QGIS Plugin zur Visualisierung und Analyse von multi- und hyperspektralen
  Fernerkundungsdaten. In der Session werden die wichtigsten Konzepte und Funktionalitäten anhand
  praktischer Beispiele live demonstriert.%
}%


%%%%%%%%%%%%%%%%%%%%%%%%%%%%%%%%%%%%%%%%%%%

% time: Thursday 09:00
% URL: https://pretalx.com/fossgis2020/talk/KS3NDA/

%

\noindent\abstractHSRundbau{%
  Jelto Buurman, Jan Klein%
}{%
  Vorbereitung einer Großveranstaltung mit QGIS und QField%
}{%
Der Rheinland-Pfalztag 2019 in Annweiler%
}{%
  Es wird die Unterstützung der Vorbereitung des Rheinland-Pfalz\-tages in Annweiler 2019 von der
  Aufnahme des Straßenmobiliars mit dem Trimble Catalyst GPS bis hin zur Erzeugung der
  Planungsunterlagen beschrieben.%
}%

%%%%%%%%%%%%%%%%%%%%%%%%%%%%%%%%%%%%%%%%%%%

% time: Thursday 09:00
% URL: https://pretalx.com/fossgis2020/talk/GYWH39/

%

\noindent\abstractWeismannhaus{%
  Sarah Hoffmann%
}{%
  OSM-Daten verarbeiten mit Python und\\ Pyosmium%
}{%
}{%
  Pyosmium bietet eine Möglichkeit OSM-Rohdaten schnell in\linebreak Python zu verarbeiten.
  Dieser Vortrag demonstriert anhand von praktischen Beispielen, wie man eigene Tools zur
  Datenaufbereitung schreiben kann. Er erklärt die Besonderheiten des
  OSM-Datenmodells und zeigt Techniken auf, wie man effizient mit den Daten umgeht.%
}%
\sponsorBoxA{403-osgeolive.pdf}{0.55\textwidth}{3}{%
\textbf{Medienpartner}\\
OSGeoLive bietet Ihnen die Möglichkeit, freie und Open"=Source"=Software in Verbindung mit Geodaten
auszuprobieren, ohne aufwendige Installationen durchführen zu müssen. Das OSGeoLive"=Projekt wird von
der OSGeo Foundation und vielen weiteren Aktiven getragen.%
}


%%%%%%%%%%%%%%%%%%%%%%%%%%%%%%%%%%%%%%%%%%%

% time: Thursday 09:30
% URL: https://pretalx.com/fossgis2020/talk/XNXHGP/

%
\newTimeslot{09:30}
\noindent\abstractHSRundbau{%
  Mirko Blinn%
}{%
  Aufbereitung von vektorbasierten Geodaten als Grundlage für Landnutzungsmodelle mit QGIS und Postgis%
}{%
}{%
  In den letzten Jahren hat die öffentliche Verfügbarkeit von vektorbasierten Geodaten stark
  zugenommen. Dieser Fundus an Daten wird zur Entwicklung von Landnutzungsmodellen kaum genutzt. Im
  Vortag wird gezeigt wie vektorbasierte Geodaten aus unterschiedlichsten Quellen mit QGIS und
  Postgis für die Verwendung in Landnutzungsmodellen aufbereitet werden und welche spannenden
  Informationen bereits während der Aufbereitung gewonnen werden können.%
}%


%%%%%%%%%%%%%%%%%%%%%%%%%%%%%%%%%%%%%%%%%%%

% time: Thursday 09:30
% URL: https://pretalx.com/fossgis2020/talk/DKP3WZ/

%

\noindent\abstractWeismannhaus{%
  Robert Klemm%
}{%
  OSM-Daten mit Vektortiles erfolgreich nutzen%
}{%
}{%
  Das OpenMapTile-Projekt bietet zahlreiche Möglichkeiten für den Umgang mit Vektor- und Rasterdaten. Speziell für den Umgang eines Kartendienstes aus OpenStreetMap-Daten werden verschiedene Lösungen bereitgestellt. Dieser Vortrag bietet einen Überblick über den Einsatz und Prozessierung bis zur Einbindung von Vektortiles in einer Webkarte.%
}%


%%%%%%%%%%%%%%%%%%%%%%%%%%%%%%%%%%%%%%%%%%%

% time: Thursday 10:00
% URL: https://pretalx.com/fossgis2020/talk/PHJ8VG/

%
\newTimeslot{10:00}
\noindent\abstractHSAnatomie{%
  Sven Bingert, Martin Weis, Christian Bauer%
}{%
  Offene Smart Farming Produkte aus offenen Satellitendaten%
}{%
}{%
  Das Projekt Open Forecast entwickelt eine generische Infrastruktur zur Verarbeitung offener Daten
  auf HPC-Systemen. Ein Anwendungsfall hat zum Ziel, Produkte für Smart Farming in der
  Landwirtschaft aus offenen Satellitendaten abzuleiten. Die Ergebnisse werden als offene Daten als
  OGC-konforme Geodatendienste verfügbar gemacht werden. Wir präsentieren in unserem Beitrag den
  aktuellen Stand der Verarbeitungskette unter Verwendung des ESA Frameworks sen2agri.%
}%


%%%%%%%%%%%%%%%%%%%%%%%%%%%%%%%%%%%%%%%%%%%

% time: Thursday 10:00
% URL: https://pretalx.com/fossgis2020/talk/DKTG8X/

%

\noindent\abstractHSRundbau{%
  Marco Hugentobler%
}{%
  Vektorverschneidung mit QGIS%
}{%
}{%
  Vektorverschneidung gehört zu den grundlegenden GIS-Analyse\-funktionen. Dennoch kommt es in der
  Praxis häufig zu unangenehmen Überraschungen, zum einen wegen numerischen Problemen der
  verwendeten Funktionen, zum anderen weil die Verschneidungsoperationen sehr rechenintensiv sein
  können. Dieser Vortrag stellt die verschiedenen Vektorverschneidungsfunktionen im
  Verarbeitungsmodul von QGIS vor und analysiert anhand von praktischen Beispielen deren Stärken und
  Schwächen.%
}%


%%%%%%%%%%%%%%%%%%%%%%%%%%%%%%%%%%%%%%%%%%%

% time: Thursday 10:00
% URL: https://pretalx.com/fossgis2020/talk/9ZPDW7/

%

\noindent\abstractWeismannhaus{%
  Sven Geggus%
}{%
  OSMPOIDB, eine kontinuierlich aktualisierte POI Datenbank auf Openstreetmap-Basis%
}{%
}{%
  Neben Routing und dem Erzeugen von Karten stellt ist die Darstellung sogenannter Points of
  Interest (POI) eine häufige Anwendung von Openstreetmap dar. Im Vortrag zeige ich wie man ein Backend für solche POI Karten einfach selbst aufsetzen und
  betreiben kann.%
}%


%%%%%%%%%%%%%%%%%%%%%%%%%%%%%%%%%%%%%%%%%%%

% time: Thursday 10:00
% URL: https://pretalx.com/fossgis2020/talk/LMZ3RU/

%

\noindent\abstractOther{%
  FOSSGIS e.V. Vorstand%
}{%
  AUF EIN WORT mit dem Vorstand%
}{%
}{%
  Wir im Vorstand wollen die Konferenz nutzen, um mit Euch, den Mitgliedern und der weiteren
  Community ins Gespräch zu kommen.%
}%
{%
  Meetingraum%
}%
\sponsorBoxA{205-disy}{0.40\textwidth}{4}{%
\textbf{Bronzesponsor}\\
Wir verbinden Datenanalytik und Geoinformation auf innovative Weise. Basis
hierfür ist unsere Kerntechnologie Cadenza. Gemeinsam mit mehr als 110
Mitarbeitern an unserem Standort in Karlsruhe arbeiten wir tagtäglich mit
Begeisterung und Kreativität an individuellen Lösungen für unsere Kunden.
}


%%%%%%%%%%%%%%%%%%%%%%%%%%%%%%%%%%%%%%%%%%%

% time: Thursday 11:00
% URL: https://pretalx.com/fossgis2020/talk/XUSWXK/

%
\newTimeslot{11:00}
\noindent\abstractHSAnatomie{%
  Marc Jansen, Manuel Fischer%
}{%
  Neuentwicklung der GDI-DE Testsuite%
}{%
}{%
  Die GDI-DE Testsuite ist die zentrale Testplattform der GDI-DE zur Prüfung der Konformität von
  Geodaten und -diensten zu nationalen und internationalen Standards. Aktuell wird die Open Source
  Software GDI-DE Testsuite in einem umfangreichen Projekt komplett neu entwickelt. Ziele der
  Neuentwicklung sind u.a. die Verbesserung der Usability, die Erhöhung der Ausfallsicherheit, die
  flexible Integration mehrerer Test-Engines und die Erweiterung des Funktionsumfangs.%
}%


%%%%%%%%%%%%%%%%%%%%%%%%%%%%%%%%%%%%%%%%%%%

% time: Thursday 11:00
% URL: https://pretalx.com/fossgis2020/talk/KU8UZS/

%

\noindent\abstractHSRundbau{%
  Torsten Friebe, Michael Schulz%
}{%
  Einsatz von XPlanung in der kommunalen\\ Praxis~-- ein Werkstattbericht%
}{%
}{%
  Die Stadt Freiburg hat sich Ende 2018 auf den Weg gemacht, die digitale Bauleitplanung für
  Bebauungsplanverfahren auf die Verwendung des XÖV Standards XPlanung umzustellen. Ein Jahr nach
  der Einführung werden die Erfahrungen und verschiedenen Schritte zur Umsetzung mit FOSS, darunter
  Werkzeuge wie deegree, HALE, QGIS und GeoNetwork, im Vortrag beschrieben sowie die Ergebnisse aber
  auch die Schwierigkeiten vorgestellt.%
}%


%%%%%%%%%%%%%%%%%%%%%%%%%%%%%%%%%%%%%%%%%%%

% time: Thursday 11:00
% URL: https://pretalx.com/fossgis2020/talk/SBUNBG/

%

\noindent\abstractWeismannhaus{%
  Rafael Hologa, Nils Riach%
}{%
  Gefahrenbewertung im Radverkehr mittels Crowdsourcing von Geoinformationen%
}{%
}{%
  Der Beitrag diskutiert das Potential von systematisch erfassten Bürgerinformationen zur Analyse
  von gesellschaftlich relevanten Raumphänomenen. Mittels Crowdsourcing von Geoinformationen zu
  Gefahren im Radverkehr in Freiburg im Breisgau und deren Verschneidung mit amtlichen
  Unfallstatistiken sowie weiteren Geodaten wird gezeigt, wie auf einer solchen Geodatenbasis Sicherheitsaspekte für Radfahrer neu bewertet werden können.%
}%
\sponsorBoxA{106-db-mindbox.pdf}{0.57\textwidth}{3}{%
\textbf{Silbersponsor}\\
GoBeta.de ist die offene Innovations"=Plattform, auf der wir uns mit
Startups, Partnern und Unterstützern austauschen, zusammenarbeiten und
neue Ideen vorstellen.

Für Entwicklerinnen, Techies, Hackerinnen und Mobilitätsbegeisterte
Tester\allowbreak*innen bieten wir eine Community"=Plattform mit Daten,
Codes, Anleitungen und Infos zu echten Projekten.

Bringt Eure Ideen ein, sucht Unterstützung und lernt Euch online und im
echten Leben kennen, z.\,B. auf Hackathons und MeetUps. Neben Zugang zu
Daten von DB und Partner könnt Ihr hier auch Feedback von
Nutzer\allowbreak*innen bekommen. Gemeinsam gestalten wir die Zukunft der
Mobilität als Teil der Digitalisierungsaktivitäten der DB. \#powered by DB
mindbox und Euch!
}


%%%%%%%%%%%%%%%%%%%%%%%%%%%%%%%%%%%%%%%%%%%

% time: Thursday 11:30
% URL: https://pretalx.com/fossgis2020/talk/HQD7NN/

%
\newTimeslot{11:30}
\noindent\abstractHSAnatomie{%
  Pirmin Kalberer%
}{%
  Der neue OGC-API-Standard ist da!%
}{%
}{%
  Was unter dem Arbeitsnamen WFS3 begann, hat sich zu einer kompletten Überarbeitung und
  Vereinheitlichung diverser vorhandener OGC-Standards entwickelt. Neu war nicht nur der offene
  Prozess, sondern auch mehrere Umsetzungen bereits während der Entwurfsphase. Das Resultat namens
  \emph{OGC API~-- Features~-- Part 1: Core} ist ein zeitgemässer und schlanker Standard, welcher zwar noch
  viele WFS-2-Funktionen vermissen lässt, aber dank der Erweiterbarkeit gute Chancen auf eine grosse
  Verbreitung hat.%
}%


%%%%%%%%%%%%%%%%%%%%%%%%%%%%%%%%%%%%%%%%%%%

% time: Thursday 11:30
% URL: https://pretalx.com/fossgis2020/talk/WTKV7Y/

%

\noindent\abstractHSRundbau{%
  Christian Strobl%
}{%
  Freie Geodaten der Umweltverwaltung Bayern%
}{%
Hintergründe und einfache Anwendungsbeispiele%
}{%
  Gemäß dem Bayerischen Umweltinformationsgesetz (Bay\-UIG) und dem Bayerischen
  Geodateninfrastrukturgesetz (Bay\-GDIG) stellt die Bayerische Umweltverwaltung Geodaten unter der
  freien Lizenz CC-BY 3.0 zur Verfügung. In dem Vortrag werden die gesetzlichen Grundlagen und
  gebräuchliche Lizenzen (Creative Commons CC, Datenlizenz Deutschland, $\ldots$) am Beispiel von
  Geodaten des Bayerischen Umweltressorts erläutert und vereinzelte Anwendungsbeispiele gegeben.%
}%


%%%%%%%%%%%%%%%%%%%%%%%%%%%%%%%%%%%%%%%%%%%

% time: Thursday 11:30
% URL: https://pretalx.com/fossgis2020/talk/BJSHXE/

%

\noindent\abstractWeismannhaus{%
  Simon Metzler%
}{%
  Entwicklung des Berliner Radverkehrs anhand von öffentlich gemachten Verkehrszähldaten%
}{%
}{%
  Vorstellung einer Webseite, die mithilfe von öffentlich gemachten Verkehrszähldaten einen Eindruck
  über die Veränderungen im Berliner (Rad-)Verkehr gibt. Dabei kann der Nutzer auf einer Karte ein
  Postleitzahlengebiet auswählen und die Zähldaten-Zeitreihen je Zählort nachvollziehen.%
}%
\sponsorBoxA{105-con4gis-small}{0.53\textwidth}{5}{%
\textbf{Silbersponsor}\\
con4gis ist der GIS"=Baukasten für das beliebte Content"=Management"=System Contao. Seit 2010 wird er in Jever von der Küstenschmiede entwickelt. Contao und con4gis sind Open Source auf GitHub.

Die aktuelle Contao LTS Version 4.9 und die neue con4gis 7 aus Februar 2020 sind gute Einstiegsmöglichkeiten in beide Systeme. con4gis umfasst eine große Bibliothek an Symfony-Bundles zum Aufbau von Webanwendungen. Die Kartentechnik nutzt OpenLayers 6 und andere etablierte Open-Source- und Open-Data-Technologien. Contao und con4gis sind aktuell dokumentiert.

Mehr zur Technik und zur aktuellen Version findet Ihr in Eurer Konferenztasche oder auf con4gis.org.
}


%%%%%%%%%%%%%%%%%%%%%%%%%%%%%%%%%%%%%%%%%%%

% time: Thursday 12:00
% URL: https://pretalx.com/fossgis2020/talk/CWRZQH/

%
\newTimeslot{12:00}
\noindent\abstractHSAnatomie{%
  Jörg Thomsen%
}{%
  MapServer Statusbericht%
}{%
}{%
  Am Horizont erscheinen die ersten Anzeichen für MapServer 8. Was wird er Neues bringen? Welche
  Neuerungen gab es in den letzten 1-2 Jahren? Vorangestellt wird eine kurze Einführung, damit auch
  Neueinsteiger dem Vortrag folgen können.%
}%


%%%%%%%%%%%%%%%%%%%%%%%%%%%%%%%%%%%%%%%%%%%

% time: Thursday 12:00
% URL: https://pretalx.com/fossgis2020/talk/CFXVAL/

%

\noindent\abstractHSRundbau{%
  Sebastian Ratjens%
}{%
  Map Editor für individuelle amtliche Vektorkarten%
}{%
}{%
  Mit dem Prototyp "`Map Editor"' veröffentlicht das Landesamt für Geoinformation und Landesvermessung
  Niedersachsen (LGLN) eine neue Open-Source-Software zur Erstellung individueller Basiskarten und
  Kartenanwendungen, basierend auf Vektortiles. Die Anwendung bietet Werkzeuge zur Anpassung und
  Veröffentlichung der Vektorkarten und ist für die Nutzung auf mobilen Endgeräten optimiert.%
}%

%%%%%%%%%%%%%%%%%%%%%%%%%%%%%%%%%%%%%%%%%%%
% Lightning Talks
%%%%%%%%%%%%%%%%%%%%%%%%%%%%%%%%%%%%%%%%%%%

% time: Thursday 13:30
% URL: https://pretalx.com/fossgis2020/talk/NHUAP9/

%
\newTimeslot{13:30}
\noindent\abstractHSAnatomie{%
  Felix Kunde%
}{%
  Schneller, besser, leichter~-- PostGIS 3%
}{%
}{%
  Seit Jahren gehört PostGIS zum Repertoire von Entwicklern und Firmen bei der Verwaltung und
  Analyse von Geodaten. Die neue Hauptversion präsentiert sich mit unzähligen Detailverbesserungen in
  Funktionalität und Geschwindigkeit, von denen die wichtigsten im Vortrag erklärt werden.%
}%


%%%%%%%%%%%%%%%%%%%%%%%%%%%%%%%%%%%%%%%%%%%

% time: Thursday 13:30
% URL: https://pretalx.com/fossgis2020/talk/8ZASBP/

%

\noindent\abstractHSRundbau{%
  Andreas Neumann%
}{%
  QGIS Kartografie-Verbesserungen 2019%
}{%
}{%
  Im Vortrag werden einige der Verbesserungen im Bereich Kartografie (Symbologie, Beschriftung,
  Kartenlayout) aus dem Jahr 2019 (Versionen 3.8 bis 3.12) vorgestellt.%
}%


%%%%%%%%%%%%%%%%%%%%%%%%%%%%%%%%%%%%%%%%%%%

% time: Thursday 13:30
% URL: https://pretalx.com/fossgis2020/talk/BYJJNM/

%

\noindent\abstractWeismannhaus{%
  Matthias Hinz%
}{%
  Geoprocessing mit OpenCaching%
}{%
}{%
  In diesem Vortrag wird gezeigt, wie die Datenbank der Geocaching-Plattform OpenCaching.de mit QGIS
  und PostGIS wissenschaftlich ausgewertet werden kann. Mithilfe des Graphical Modeler von QGIS, SQL
  und Python können die meisten Arbeitsschritte automatisiert werden. Die Analyse ist ein
  Best-Practice-Beispiel zum Umgang mit offenen Geodaten im Rahmen des BMVI-geförderten Projektes
  OpenGeoEdu. Sie kann auf der offenen Lernplattform des Projektes ausführlich nachvollzogen werden.%
}%


%%%%%%%%%%%%%%%%%%%%%%%%%%%%%%%%%%%%%%%%%%%

% time: Thursday 14:00
% URL: https://pretalx.com/fossgis2020/talk/ZP3JZZ/

%
\newTimeslot{14:00}
\noindent\abstractHSAnatomie{%
  Astrid Emde%
}{%
  Verbindungen schaffen mit PostgreSQL\\ Foreign Data Wrappern%
}{%
}{%
  Über Foreign Data Wrapper können Verbindungen aus einer\linebreak PostgreSQL-Datenbank heraus zu anderen
  externen Quellen aufgebaut werden.
 Dadurch müssen sich nicht mehr alle Daten, die in einem Projekt verwendet werden, auch innerhalb
  der\linebreak PostgreSQL-Datenbank befinden.
  Mit unterschiedlichen Erweiterungen können Verbindungen von PostgreSQL zu Oracle, MySQL, CSV,
  JSON, Geodaten, OSM und vielen weiteren Quellen geschaffen werden.%
}%


%%%%%%%%%%%%%%%%%%%%%%%%%%%%%%%%%%%%%%%%%%%

% time: Thursday 14:00
% URL: https://pretalx.com/fossgis2020/talk/VDPQS7/

%

\noindent\abstractHSRundbau{%
  Dirk Stenger%
}{%
  TEAM Engine%
}{%
 Vorstellung der neusten Tests für OGC-Standards wie \emph{OGC API~--~Features} oder GeoTIFF%
}{%
  Die TEAM Engine ist eine Engine, mit der Entwickler und Anwender Geodienste wie WMS und
  Geoformate wie GeoPackage testen können.
 Es werden aktuell mehrere neue Testsuites entwickelt, mit denen unter anderem Implementierungen
  der neuen OGC-Standards \emph{OGC API~--~Features} und GeoTIFF getestet werden können.
  Dieser Vortrag gibt eine Übersicht über alle neuen Testsuites und stellt einige von ihnen näher
  vor. Des Weiteren werden die aktuellen Entwicklungen im TEAM Engine Projekt aufgezeigt.%
}%


%%%%%%%%%%%%%%%%%%%%%%%%%%%%%%%%%%%%%%%%%%%

% time: Thursday 14:00
% URL: https://pretalx.com/fossgis2020/talk/QJBX8R/

%

\noindent\abstractWeismannhaus{%
  Svenja Ruthmann, Alexander Rolwes, Klaus Böhm%
}{%
  Räumliche Verortung von textbasierten Social-Media-Einträgen am Beispiel von \mbox{Polizei-Tweets}%
}{%
}{%
  Ziel der Forschungsinitiative ist die Untersuchung der räumlichen Verortung deutscher Tweets auf
  Basis von verfügbaren Standardwerkzeugen. Zunächst werden die spezifischen Herausforderungen im
  Kontext der Sprache und der Eigenschaften von Tweets betrachtet. Die darauffolgende Entwicklung
  eines grundlegenden, algorithmischen Ablaufs schließt mit der Implementierung eines Prototyps ab.
  Dieser bildet die Basis für die Evaluation der Genauigkeit und führt zur Betrachtung des
  Verbesserungspotentials.%
}%

\sponsorBoxA{203-geoinformatik-buero-dassau}{0.26\textwidth}{2}{%
\textbf{Bronzesponsor}\\
Die Geoinformatikbüro Dassau GmbH aus Düsseldorf bietet seit 2006
Beratung, Konzeption, Schulung, Wartung, Support und Programmierung zum
Thema GIS und GDI auf Open-Source-Basis. Ein Fokus liegt auf der Software
QGIS, QGIS Server, QGIS Web Client, GBD WebSuite, Postgres/PostGIS und
GRASS GIS.
}

%%%%%%%%%%%%%%%%%%%%%%%%%%%%%%%%%%%%%%%%%%%

% time: Thursday 14:30
% URL: https://pretalx.com/fossgis2020/talk/UPPG8U/

%
\newTimeslot{14:30}
\noindent\abstractHSAnatomie{%
  Michael Schulz%
}{%
  AD und PostgreSQL-Rollen verknüpfen mit dem Höllenhund%
}{%
}{%
  Rollen und Rechte in einer Geodatenbank zu verwalten, war schon immer eine Herkulesaufgabe. In
  einer Stadtverwaltung wie Freiburg, mit mehreren hundert Mitarbeitenden aus vielen verschiedenen
  Ämtern, bietet es sich daher an, diese Berechtigungen über die zentrale Benutzerverwaltung zu
  steuern. Das Zusammenspiel von FOSSGIS-Komponenten mit dem Verzeichnisdienst ActiveDirectory wird
  mit Kerberos als Authentifizierungsdienst umgesetzt.%
}%


%%%%%%%%%%%%%%%%%%%%%%%%%%%%%%%%%%%%%%%%%%%

% time: Thursday 14:30
% URL: https://pretalx.com/fossgis2020/talk/WNARKE/

%

\noindent\abstractHSRundbau{%
  Torsten Friebe, Dirk Stenger%
}{%
  OSGeo-Projekt deegree 2020%
}{%
Neuigkeiten zu \emph{OGC API~--~Features}%
}{%
  Das OSGeo-Projekt deegree stellt seit mehreren Jahren umfassende Referenzimplementierungen für
  OGC-Webservices wie z.B. WFS, WMS und WMTS bereit. Der Vortrag geht auf die aktuellen Neuerung in
  deegree und die Unterstützung des neue Standards \emph{OGC API~--~Features} ein.%
}%


%%%%%%%%%%%%%%%%%%%%%%%%%%%%%%%%%%%%%%%%%%%
% Lightning Talks
%%%%%%%%%%%%%%%%%%%%%%%%%%%%%%%%%%%%%%%%%%%

% time: Thursday 15:00
% URL: https://pretalx.com/fossgis2020/talk/7HVALE/

%%%%%%%%%%%%%%%%%%%%%%%%%%%%%%%%%%%%%%%%%%%

% time: Thursday 15:30
% URL: https://pretalx.com/fossgis2020/talk/MNKHF8/

%
\newTimeslot{15:30}
\noindent\abstractHSAnatomie{%
  Daniel Koch, Carmen Tawalika%
}{%
  FOSS in der Cloud%
}{%
}{%
  Mit den täglich wachsenden Geodatenpools steigen Anforderungen an Hard- und Software. Lokale
  Installationen sind in den letzten Jahren vermehrt durch verteilte Systeme in der Cloud abgelöst
  worden, welche sich auch im Bereich der FOSS etabliert haben.
 Dieser Vortrag gibt einen Überblick über Konzepte des Cloud Computings, der verfügbaren Tools und
  Voraussetzungen für eine erfolgreiche Skalierbarkeit, sowie Erfahrungen und Empfehlungen am
  Beispiel von actinia, GRASS GIS, GeoServer und SHOGun.%
}%


%%%%%%%%%%%%%%%%%%%%%%%%%%%%%%%%%%%%%%%%%%%

% time: Thursday 15:30
% URL: https://pretalx.com/fossgis2020/talk/CTJEVS/

%

\noindent\abstractHSRundbau{%
  Marc Jansen, Andreas Hocevar%
}{%
  OpenLayers: v6.x und wie es weitergeht%
}{%
}{%
  Im Vortrag werden der aktuelle Stand, die relevantesten Änderungen und potentielle künftige
  Weiterentwicklungen der weit verbreiteten JavaScript-Bibliothek OpenLayers vorgestellt.%
}%
\sponsorBoxA{208-geofabrik}{0.53\textwidth}{3}{%
  \textbf{Bronzesponsor}\\
  Die Geofabrik bietet die Datenaufbereitung von OpenStreetMap"=Daten an,
  betreibt OpenStreetMap-basierte Serverdienste und hilft Ihnen bei der
  Installation eigener Karten-, Geocoding- oder Routingserver.
}


%%%%%%%%%%%%%%%%%%%%%%%%%%%%%%%%%%%%%%%%%%%

% time: Thursday 15:30
% URL: https://pretalx.com/fossgis2020/talk/TUTT33/

%

\noindent\abstractWeismannhaus{%
  Arndt Brenschede, Norbert Renner%
}{%
  Routenplanung mit BRouter und BRouter-Web%
}{%
}{%
  Wir geben einen kurzen Überblick der OSM-basierten Routing-Engine BRouter selbst sowie der
  populärsten Arten, sie zu benutzen. Neben mobilen Karten- und Navigations-Apps wie OsmAnd,
  OruxMaps und Locus Maps ist das insbesondere die Web-Anwendung "`BRouter-Web"', deren Funktionen und
  Besonderheiten wir vorstellen.%
}%
\sponsorBoxA{104-regiodata}{0.47\textwidth}{5}{%
\textbf{Silbersponsor, Aussteller}\\
regioDATA bietet als Dienstleister komplette Lösungen rund um
Geoinformationssysteme (GIS) für Ver- und Entsorgungsnetze, sowie die
Planung und Bauleitung für Energieversorger und Kommunen an.  Raumbezogene
Informationssysteme sind seit 1997 das Spezialgebiet des Unternehmens.
Umfassender Service, Ingenieurvermessung, lückenlose Dokumentation und
Planungsleistungen, sowie die Software-Entwicklung sind die Eckpfeiler der
Kundenbetreuung.  Eine ausgereifte Produktfamilie mit Webbasierten
Kartendiensten und Applikationen für Fachanwender runden das Angebot ab.
regioDATA ist Teil der badenovaGRUPPE mit derzeit ca. 130 Mitarbeitenden.
}


%%%%%%%%%%%%%%%%%%%%%%%%%%%%%%%%%%%%%%%%%%%

% time: Thursday 16:00
% URL: https://pretalx.com/fossgis2020/talk/KNUBLU/

%
\newTimeslot{16:00}
\noindent\abstractHSAnatomie{%
  Markus Neteler, Carmen Tawalika%
}{%
  GRASS GIS in der Cloud%
}{%
actinia Geoprozessierung
}{%
  Ursprünglich GRaaS (GRASS as a Service) genannt, wurde actinia entwickelt, um GRASS GIS
  Funktionalität über eine HTTP REST API bereitzustellen. GRASS GIS Locations, Mapsets und Geodaten
  werden zu Ressourcen, die per REST verwaltet und visualisiert werden können. Actinia folgt dem
  Paradigma, Algorithmen zu Clouddaten zu bringen und unterstützt u.a. persistente und flüchtige
  Berechnung, Benutzerverwaltung zur Begrenzung von Pixeln, Prozessen und Berechnungsdauer.%
}%


%%%%%%%%%%%%%%%%%%%%%%%%%%%%%%%%%%%%%%%%%%%

% time: Thursday 16:00
% URL: https://pretalx.com/fossgis2020/talk/GHET3U/

%

\noindent\abstractHSRundbau{%
  Jakob Miksch, Christian Mayer%
}{%
  Wegue~-- OpenLayers und Vue.js in der Praxis%
}{%
}{%
  Mit Wegue lassen sich WebGIS-Anwendungen durch eine einfache Konfigurationsdatei erstellen. Es
  beinhaltet bereits gängige Funktionenen wie Layerswitcher oder Geocoding. Durch die modulare
  Struktur kann Wegue jedoch auch leicht erweitert und individuell angepasst werden.%
}%

\pagebreak
%%%%%%%%%%%%%%%%%%%%%%%%%%%%%%%%%%%%%%%%%%%

% time: Thursday 16:00
% URL: https://pretalx.com/fossgis2020/talk/EMQGW7/

%

\noindent\abstractWeismannhaus{%
  Toni Erdmann%
}{%
  PTNA%
}{%
 Qualitätssicherung für ÖPNV in OpenStreetMap
}{%
  ÖPNV Informationen werden in OSM mittels Route-Relationen modelliert.
  Existierende  Qualitätssicherungstools analysieren zumeist einzelne Route-Relationen.
 \emph{PTNA~-- Public Transport Network Analysis} erlaubt zusätzlich im ersten Schritt eine
  Soll-Ist-Analyse bezüglich der von einem Verkehrsverbund angebotenen Linien und den in OSM
  gemappten Route-Relationen.
  Im zweiten Schritt werden die vorhanden Route-Relationen einer Fehleranalyse unterzogen.
  PTNA wird in diesem Vortrag vorgestellt.%
}%

\sponsorBoxA{204-gkg-kassel}{0.40\textwidth}{6}{%
\textbf{Bronzesponsor, Aussteller}\\
GKG Kassel, Dr.-Ing. Claas Leiner~--
Schulungen, Dienstleistungen und Sup\-port rund um QGIS, GRASS, Spa\-tia\-Lite und PostGIS. Mit
angepassten QGIS"=Oberflächen, dem QGIS"=Modeller sowie SpatialSQL strukturiere ich Ihre Geodaten und
ermögliche maßgeschneiderte Analyse und Präsentation mit freier Software.%
}

%%%%%%%%%%%%%%%%%%%%%%%%%%%%%%%%%%%%%%%%%%%

% time: Thursday 16:30
% URL: https://pretalx.com/fossgis2020/talk/MWDS79/

%
\newTimeslot{16:30}
\noindent\abstractHSAnatomie{%
  pebau%
}{%
  Das OSGeo Datacube Community Project\\ \emph{rasdaman}%
}{%
}{%
  Die \emph{rasdaman} ("raster data manager") Datacube Engine ist OGC-Referenzimplementierung für den
  Rasterdaten-Standard WCS, mit über 28.000 Downloads von \emph{open-source rasdaman community}. Wir
  stellen den aktuellen Stand des Systems vor und präsentieren Portale, welche \emph{rasdaman community}
  nutzen.%
}%


%%%%%%%%%%%%%%%%%%%%%%%%%%%%%%%%%%%%%%%%%%%

% time: Thursday 16:30
% URL: https://pretalx.com/fossgis2020/talk/EBMZWN/

%

\noindent\abstractHSRundbau{%
  Michael Kahle%
}{%
  javascript-Bibliotheken zur Einbindung von historischen Umwelt- und Klimainformationen als Kartenlayer%
}{%
}{%
  \emph{www.tambora.org} bietet über 250000 historische Einträge zum Thema Umwelt- und Klima,
  differenzierbar nach Zeit, Ort und thematischen Schwerpunkt. Durch die vorgestellten Libraries
  lassen sich diese einfach in eigene leaflet- oder openlayers-Karten  einbinden und reichern
  diese so durch wertvolle Informationen zur räumlichen und zeitlichen Verteilung vergangener
  Ereignisse an.%
}%

\pagebreak
%%%%%%%%%%%%%%%%%%%%%%%%%%%%%%%%%%%%%%%%%%%

% time: Thursday 16:30
% URL: https://pretalx.com/fossgis2020/talk/VAXSHT/

%

\noindent\abstractWeismannhaus{%
  Patrick Brosi%
}{%
  Automatische Korrektur von ÖV-Stationen in OSM%
}{%
}{%
  ÖV-Stationen bestehen üblicherweise aus mehreren Punkt-, Linien- oder Flächeninformationen, z.B.
  Haltepunkten, Gleisen oder Bahnhofsgebäuden u.a.. Diese Objekte können in OSM mittels
  übergeordneter Relationen verknüpft werden, allerdings fehlt diese Gruppierung häufig oder ist
  unvollständig. Wir stellen ein Tool vor, das die Elemente von ÖV-Stationen als Paare von
  Stationsnamen und -koordinaten abstrahiert und mittels Ähnlichkeitsmaßen und maschinellem Lernen
  OSM-Stationen korrigieren kann.%
}%

\sponsorBoxA{202-wagner-it-small}{0.35\textwidth}{3}{%
\textbf{Bronzesponsor}\\
Wagner-IT ist seit 2011 für kleinere Gemeinden und Städte im Bereich der
GIS- und Web-GIS"=Betreuung sowie des Geodatenmanagements tätig.
Realisiert werden individuelle Lösung auf Basis von QGIS, QGIS-Server,
PostgreSQL und Web-Clients (z.\,B. dem Lizmap-Client). https://wagner-it.de/
}

%%%%%%%%%%%%%%%%%%%%%%%%%%%%%%%%%%%%%%%%%%%

% time: Thursday 17:00
% URL: https://pretalx.com/fossgis2020/talk/M73SCJ/

%
\newTimeslot{17:00}
\noindent\abstractHSRundbau{%
  Christian Mayer, Jan Suleiman%
}{%
  Neues vom GeoStyler%
}{%
}{%
  GeoStyler ist eine Open-Source-JavaScript-Bibliothek zur einfachen Erstellung von modernen
  Web-Oberflächen zum kartographischen Stylen von Geodaten. Somit wird es dem Anwender möglich ohne
  Programmierung und Editieren von Text-Dateien (XML und co.) Styling-Vorschriften interaktiv zu
  gestalten und diese in diverse offene Style-Formate zu überführen. Der Vortrag stellt die
  Neuerungen des letzten Jahres vor und gibt einen Überblick über zukünftige Entwicklungen im
  GeoStyler-Projekt.%
}%


%%%%%%%%%%%%%%%%%%%%%%%%%%%%%%%%%%%%%%%%%%%

% time: Thursday 17:00
% URL: https://pretalx.com/fossgis2020/talk/UKFLSU/

%

\noindent\abstractWeismannhaus{%
  Leoni Möske%
}{%
  Qualitätsbewertung von OpenStreetMap-Gebäudedaten%
}{%
 Am Beispiel der Stadtgebiete Köln und Gera%
}{%
  Der Vortrag geht vor allem auf die Entstehung der heterogenen Datenqualität in OpenStreetMap und
  die Umsetzung der intrinsischen und extrinsischen Qualitätsbewertungsmethode ein. Die Methoden
  bewerten die OSM-Datenqualität basierend auf der OSM-Datenhistorie (intrinsisch) und im Vergleich
  zu amtlichen ALKIS-Gebäudedatensätzen (extrinsisch).%
}%


%%%%%%%%%%%%%%%%%%%%%%%%%%%%%%%%%%%%%%%%%%%

% time: Thursday 18:00
% URL: https://pretalx.com/fossgis2020/talk/PH3E8X/

%
\newTimeslot{18:00}
\noindent\abstractWeismannhaus{%
%
}{%
  Mitgliederversammlung FOSSGIS e.V.%
}{%
}{%
  Zur jährlich stattfindenden Versammlung des FOSSGIS e.V. sind alle Mitglieder herzlich eingeladen,
  teilzunehmen und sich zu beteiligen. Einige Themen stehen auf der Agenda. Wir laden ein zum
  Kennenlernen, zur Diskussion, Abstimmung und Neuwahlen.%
}%
\noindent\sponsorBoxA{103-geoinfo}{0.38\textwidth}{2}{%
\textbf{Silbersponsor}\\
\noindent Die GEOINFO IT AG verbindet die Leidenschaft für
Geoinformationen mit der Begeisterung für Softwaretechnologien. Aus den
Ideen unserer Kunden entstehen innovative Lösungen. Darunter
bedürfnisgerechte Fachanwendungen für Infrastruktur, Sicherheit,
Vegetation und Landwirtschaft. Wir machen umfangreiche Geoinformationen
einfach und zielgruppengerecht zugänglich.

Das Rückgrat unserer Geodateninfrastrukturen bilden eigene moderne
Rechenzentren. Als einer der einzigen Schweizer Anbieter der Branche
entwickeln wir zudem selbständig Software-Produkte. Auch deswegen gehört
die GEOINFO Applications AG in der Schweiz zu den führenden Anbietern
umfassender Geodateninfrastrukturen.%
}


%%%%%%%%%%%%%%%%%%%%%%%%%%%%%%%%%%%%%%%%%%%
